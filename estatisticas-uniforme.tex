\subsection{Estatísticas para distribuição uniforme de árvores binárias}

Para árvores binárias uniformemente distribuídas,
a análise é consideravelmente mais complexa,
portanto será apenas enunciada.

O comprimento total dos caminhos internos é
em média,
\begin{equation*}
    \frac{(n+1) 4^n}{\binom{2n}{n}} \sim n \sqrt{\pi n}
\end{equation*}
para uma árvore com $n$ nós~\cite[p.~288]{SedgewickFlajolet2013}.
Portanto,
a profundidade média de um nó é $\sqrt{\pi n}$,
que é significativamente maior do que a profundidade média $2 \ln n$
para árvores de busca binárias.

E, assim como no caso das árvores de busca binárias,
a altura média de uma árvore binária gerada uniformemente
é proporcional a altura média dos nós, $\sqrt{\pi n} + O(1)$
\cite[p.~306]{SedgewickFlajolet2013}.
