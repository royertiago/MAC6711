\subsection{Quantidade de árvores binárias}
\label{sec:contagem}

Sempre que falarmos em gerar algum objeto ``aleatoriamente'',
precisamos discutir qual é a distribuição que estamos gerando.
Para gerar uma árvore binária com $n$ elementos,
podemos, por exemplo,
gerar uma permutação aleatória em $S_n$,
e então convertê-la para uma árvore binária de busca.
Entretanto, desta forma,
não geraremos todas as árvores com $n$ nós de maneira uniforme.
Por exemplo,
para $3$ nós,
ambas as permutações $(2, 1, 3)$ e $(2, 3, 1)$ geram a árvore~%
\tikz \filldraw (0, 0) circle (1pt) -- (.2, .2) circle (1pt) -- (.4, 0) circle (1pt);,
enquanto que apenas a permutação $(1, 2, 3)$ gera a árvore~%
\tikz \filldraw (0, 0) circle (1pt) -- (.1, .1) circle (1pt) -- (.2, .2) circle (1pt);.

Portanto,
concluímos que o número de árvores com $n$ nós
é menor do que o número de permutações com $n$ elementos.

Chame de $C_n$ o número de árvores binárias com $n$ nós;
vamos montar uma relação de recorrência para $C_n$.\footnote{
    Essa derivação foi baseada em~\cite[p.~125]{SedgewickFlajolet2013}.
}

Temos $C_0 = 1$, pois existe uma única árvore sem nós --- a árvore vazia\footnote{
    Se levarmos em consideração os nós externos da árvore,
    esta árvore é composta de um único nó externo.
}.
Para $n > 0$,
podemos separar as árvores pelo tamanho da subárvore esquerda.
Fixado um tamanho $k$ para a subárvore esquerda,
existem $C_{k}$ formas de gerar a subárvore esquerda
e $C_{n-k-1}$ formas de gerar a subárvore direita.
Como estas formas são independentes,
existem $C_{k} C_{n-k-1}$ árvores de $n$ nós
cuja subárvore esquerda possui $k$ nós.
Somando esses valores, obtemos a recorrência
\begin{equation*}
    C_n = \sum_{k = 0}^{n-1} C_k C_{n-k-1},
\end{equation*}
ou, equivalentemente,
\begin{equation*}
    C_{n+1} = \sum_{k = 0}^n C_{k} C_{n-k}.
\end{equation*}

O somatório à direita desta equação é uma convolução,
o que sugere que possamos usar funções geradoras
para extrair uma fórmula fechada para $C_n$.

Seja $f$ a função geradora associada.
Temos
\begin{align*}
    f(x)^2 &= \left(\sum_{k \geq 0} C_k x^k\right)^2 \\
           &= \sum_{t \geq 0} x^t \left( \sum_{k = 0}^t C_k C_{t-k} \right) \\
           &= \sum_{t \geq 0} x^t C_{t+1}
            & \text{usando a recorrência}\\
           &= \sum_{t \geq 1} x^{t-1} C_t \\
           &= \frac 1 x \left( -1 + \sum_{t \geq 0} C_t x^t \right) \\
           &= \frac{f(x) - 1}{x}. \\
    x^2 f(x)^2 &= xf(x) - x \\
    \left(xf(x) - \frac 1 2\right)^2 &= \frac 1 4 - x.
\end{align*}
Se $x$ estiver próximo de zero,
então $xf(x) - 1/2$ será próximo de $-1/2$,
portanto negativo.
Assim, teremos
\begin{align*}
    \frac 1 2 - xf(x) &= \sqrt{\frac 1 4 - x} \\
    xf(x) &= \frac 1 2 - \sqrt{\frac 1 4 - x} \\
          &= \frac 1 2 (1 - (1 - 4x)^{1/2}).
\end{align*}
Agora, podemos expandir o termo $(1 - 4x)^{1/2}$
usando a série binomial $(1 + u)^\alpha$ para $\alpha = 1/2$~\cite[p.~487]{Spivak1994}
e $u = (-4x)$,
obtendo
\begin{align*}
    xf(x) &= \frac 1 2 \left( 1 - \sum_{k \geq 0} \binom{1/2}{k} (-4x)^k \right) \\
          &= -\frac 1 2 \sum_{k \geq 1} \binom{1/2}{k} (-4x)^k \\
    f(x) &= 2 \sum_{k \geq 0} \binom{1/2}{k+1} (-4x)^k
\end{align*}
Trabalhando o coeficiente $\binom{1/2}{k+1}$, temos
\begin{align*}
    \binom{1/2}{k+1} &= \frac{
        \frac 1 2 \cdot (\frac 1 2 - 1) \cdot (\frac 1 2 - 2) \cdots (\frac 1 2 - k)
    }{ (k+1)! } & \text{pela definição de $\textstyle \binom \alpha k$}\\
    &= (-1)^k \frac{
        \frac 1 2 \cdot (1 - \frac 1 2) \cdot (2 - \frac 1 2) \cdots (k - \frac 1 2)
    }{ (k+1)! } \\
    &= (-1)^k \frac{
        1 \cdot (2 - 1) \cdot (4 - 1) \cdots (2k - 1)
    }{ 2^{k + 1} (k + 1)! } \\
    &= (-1)^k \frac{ 1 \cdot 1 \cdot 3 \cdot 5 \cdots (2k - 1) }{2^{k+1} (k+1)!}
        \frac{2 \cdot 4 \cdot 6 \cdots 2k}{2^k k!} \\
    &= (-1)^k \frac{ (2k)! }{2^{2k+1} (k+1)! k!} \\
    &= \frac{(-1)^k}{2 \cdot (k+1) \cdot 4^k} \binom{2k}{k}.
\end{align*}
Substituindo de volta na equação, temos
\begin{align*}
    f(x) &= 2\sum_{k \geq 0}
        \frac{(-1)^k}{2*(k+1) \cdot 4^k} \binom{2k}{k} (-1)^k 4^k x^k \\
        &= \sum_{k \geq 0} \frac 1 {k+1} \binom{2k} k x^k.
\end{align*}
Usando, por exemplo,
a aproximação de Stirling,
podemos mostrar que
\begin{align*}
    \binom{2k} k \sim \frac{4^k}{\sqrt{\pi k}};
\end{align*}
isto é, o limite da razão entre essas duas quantidades tende a $1$
quando $k$ tende ao infinito.
Portanto, a última série de potências converge para todo $x$ com $|x| < 1/4$.
Assim, se fizermos as contas de trás para frente,
todas essas séries de potências convergem.
Concluímos que a derivação está certa.

Provamos, portanto, a seguinte proposição.
\begin{proposition}
    Existem $C_n$ diferentes árvores binárias com $n$ nós,
    em que
    \begin{equation}
        C_n = \frac{1}{n + 1} \binom{2n}{n}
        \label{eq:catalan}
    \end{equation}
    são os números de Catalão.
\end{proposition}
