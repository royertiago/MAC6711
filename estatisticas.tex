\section{Propriedades Estatísticas de Árvores Binárias Aleatórias}

\subsection{Altura média de uma árvore de busca binária}

(A demonstração construída nesta seção é adaptada do livro de Cormen et al.%
~\cite[p.~300]{CormenLeisersonRivestStein2009}.)

Na construção de uma árvore binária de busca
a partir de uma permutação de $(1, \dots, n)$,
após adicionamos o primeiro número como a raíz,
a adição dos próximos elementos colocará
todos os números menores que a raíz na subárvore à esquerda
e todos os números maiores que a raíz na subárvore à direita.
Na análise, portanto,
precisaremos particionar uma permutação nestes dois subconjuntos.
Assim,
antes de começarmos a conta,
criaremos uma notação para esta operação
e provaremos dois lemas que serão usados para facilitar sua manipulação.
