\section{Propriedades Estatísticas de Árvores Binárias Aleatórias}

\subsection{Altura média de uma árvore de busca binária}

(A demonstração construída nesta seção é adaptada do livro de Cormen et al.%
~\cite[p.~300]{CormenLeisersonRivestStein2009}.)

Na construção de uma árvore binária de busca
a partir de uma permutação de $(1, \dots, n)$,
após adicionamos o primeiro número como a raíz,
a adição dos próximos elementos colocará
todos os números menores que a raíz na subárvore à esquerda
e todos os números maiores que a raíz na subárvore à direita.
Na análise, portanto,
precisaremos particionar uma permutação nestes dois subconjuntos.
Assim,
antes de começarmos a conta,
criaremos uma notação para esta operação
e provaremos dois lemas que serão usados para facilitar sua manipulação.

Por exemplo, se a permutação $\sigma$ corresponde a lista $(4, 5, 3, 6, 1, 2)$,
ao construírmos uma árvore a partir de $\sigma$,
$\sigma(1) = 4$ será a raíz,
a subárvore à esquerda será a árvore construída a partir da lista $(3, 1, 2)$,
e a subárvore à direita, da lista $(5, 6)$.
Como o valor dos nós da árvore não é relevante (apenas seu valor relativo é),
podemos supor que a subárvore à direita foi gerada pela lista $(1, 2)$,
que é a mesma lista,
mas ``normalizada'' para o intervalo $[1, 2]$.
Chamaremos a lista $(3, 1, 2)$,
a parte que vai para a esquerda, de $\sigma_<$
(é a parte de $\sigma$ que é menor que $\sigma(1)$, a raíz),
e a lista $(1, 2)$,
a que foi gerada a partir da lista $(5, 6)$, de $\sigma_>$.

A definição a seguir formaliza estes nomes.

\begin{definition}
    Seja $\sigma \in S_n$ uma permutação.
    Defina $\sigma_<$ e $\sigma_>$
    como sendo a sequência dos elementos menores e maiores que $\sigma(1)$
    (o primeiro elemento)
    respectivamente,
    normalizados para os intervalo $\{1, \sigma(1) - 1\}$ e $\{1, n - \sigma(1)\}$,
    respectivamente.
    Isto é,
    se $\{a_i\}_{i = 1}^{\sigma(1)-1}$ é a sequência crescente
    dos índices satisfazendo $\sigma(a_i) < \sigma(1)$,
    então $\sigma_< \in S_{\sigma(1) - 1}$
    é a permutação que satisfaz $\sigma_<(i) = \sigma(a_i)$ para todo $i$;
    e se $\{b_i\}_{i = 1}^{n - \sigma(1)}$ é a sequência crescente
    dos índices satisfazendo $\sigma(b_i) > \sigma(1)$,
    então $\sigma_> \in S_{n - \sigma(1)}$
    é a permutação que satisfaz $\sigma_>(i) = \sigma(b_i) - \sigma(1)$ para todo $i$.
\end{definition}

A normalização que $\sigma_>$ sofre garante que
tanto $\sigma_<$ quanto $\sigma_>$ sejam permutações.
Como usaremos ambas,
é importante sabermos como converter somatórios sobre um tipo no outro.

\begin{lemma}
    Chame de $\mathcal S$ o conjunto de todas as permutações.
    Se $f: \mathcal S \to \mathbb N$ é uma função qualquer,
    então
    \begin{equation}
        \sum_{\sigma \in S_n} f(\sigma_<) = \sum_{\sigma \in S_n} f(\sigma_>).
        \label{eq:lower-permutation-to-upper}
    \end{equation}
\end{lemma}

\begin{proof}
    Se $\sigma \in S_n$,
    defina $\overline \sigma \in S_n$
    (o ``conjugado'' de $\sigma$)
    por $\overline \sigma(i) = n + 1 - \sigma(i)$.
    Esta operação é bijetora
    e inverte a relação de ordem entre os elementos de $\sigma$.
    Dessa forma, os elementos maiores que $\sigma(1)$, em $\sigma$,
    se tornarão menores que $\overline \sigma(1)$, em $\overline \sigma$.
    Portanto,
    os elementos de $(\overline \sigma)_<$ são,
    exatamente,
    os conjugados dos elementos de $\sigma_>$;
    portanto, temos
    \begin{equation}
        (\overline \sigma)_< = \overline{\sigma_>}.
        \label{eq:conjugate-permutation}
    \end{equation}
    Portanto,
    \begin{align*}
        \sum_{\sigma \in S_n} f(\sigma_<) &= \sum_{\sigma \in S_n} f((\overline \sigma)_<)
        &\text{pois $\sigma \mapsto \overline \sigma$ é uma bijeção} \\
        &= \sum_{\sigma \in S_n} f(\overline{\sigma_>})
        &\text{pela equação~\ref{eq:conjugate-permutation}} \\
        &= \sum_{\sigma \in S_n} f(\sigma_>)
        &\text{pois $\sigma \mapsto \overline \sigma$ é uma bijeção.}
    \end{align*}
\end{proof}
