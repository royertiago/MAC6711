\section{Propriedades Estatísticas de Árvores Binárias Aleatórias}

\subsection{Altura média de uma árvore de busca binária}

Na construção de uma árvore binária de busca
a partir de uma permutação de $(1, \dots, n)$,
após adicionamos o primeiro número como a raíz,
a adição dos próximos elementos colocará
todos os números menores que a raíz na subárvore esquerda
e todos os números maiores que a raíz na subárvore direita.
Na análise, portanto,
precisaremos particionar uma permutação nestes dois subconjuntos.
Assim,
antes de começarmos a conta,
criaremos uma notação para esta operação
e provaremos dois lemas que serão usados para facilitar sua manipulação.

Por exemplo, ao construirmos uma árvore
a partir da permutação $\sigma = (4, 5, 3, 6, 1, 2)$,
$\sigma(1) = 4$ será a raíz,
a subárvore esquerda será a árvore construída a partir da lista $(3, 1, 2)$,
e a subárvore direita, da lista $(5, 6)$.
Como o valor dos nós da árvore não é relevante (apenas seu valor relativo é),
podemos supor que a subárvore direita foi gerada pela lista $(1, 2)$,
que é a mesma lista,
mas ``normalizada'' para o intervalo $[1, 2]$.
Chamaremos a lista $(3, 1, 2)$,
a parte que vai para a esquerda, de $\sigma_<$
(é a parte de $\sigma$ que é menor que $\sigma(1)$, a raíz),
e a lista $(1, 2)$,
a que foi gerada a partir da lista $(5, 6)$, de $\sigma_>$.

A definição a seguir formaliza estes nomes.

\begin{definition}
    Seja $\sigma \in S_n$ uma permutação.
    Defina $\sigma_<$ e $\sigma_>$
    como sendo a sequência dos elementos menores e maiores que $\sigma(1)$
    (o primeiro elemento)
    respectivamente,
    normalizados para os intervalo $\{1, \sigma(1) - 1\}$ e $\{1, n - \sigma(1)\}$,
    respectivamente.
    Isto é,
    se $\{a_i\}_{i = 1}^{\sigma(1)-1}$ é a sequência crescente
    dos índices satisfazendo $\sigma(a_i) < \sigma(1)$,
    então $\sigma_< \in S_{\sigma(1) - 1}$
    é a permutação que satisfaz $\sigma_<(i) = \sigma(a_i)$ para todo $i$;
    e se $\{b_i\}_{i = 1}^{n - \sigma(1)}$ é a sequência crescente
    dos índices satisfazendo $\sigma(b_i) > \sigma(1)$,
    então $\sigma_> \in S_{n - \sigma(1)}$
    é a permutação que satisfaz $\sigma_>(i) = \sigma(b_i) - \sigma(1)$ para todo $i$.
\end{definition}

A normalização que $\sigma_>$ sofre garante que
tanto $\sigma_<$ quanto $\sigma_>$ sejam permutações.
Como usaremos ambas,
é importante sabermos como converter somatórios sobre um tipo no outro.

\begin{lemma}
    Se $f: \mathcal S \to \mathbb N$ é uma função qualquer,
    então
    \begin{equation}
        \sum_{\sigma \in S_n} f(\sigma_<) = \sum_{\sigma \in S_n} f(\sigma_>).
        \label{eq:lower-permutation-to-upper}
    \end{equation}
\end{lemma}

\begin{proof}
    Se $\sigma \in S_n$,
    defina $\overline \sigma \in S_n$
    (o ``conjugado'' de $\sigma$)
    por $\overline \sigma(i) = n + 1 - \sigma(i)$.
    Esta operação é bijetora
    e inverte a relação de ordem entre os elementos de $\sigma$.
    Dessa forma, os elementos maiores que $\sigma(1)$, em $\sigma$,
    se tornarão menores que $\overline \sigma(1)$, em $\overline \sigma$.
    Portanto,
    os elementos de $(\overline \sigma)_<$ são,
    exatamente,
    os conjugados dos elementos de $\sigma_>$;
    portanto, temos
    \begin{equation}
        (\overline \sigma)_< = \overline{\sigma_>}.
        \label{eq:conjugate-permutation}
    \end{equation}
    Portanto,
    \begin{align*}
        \sum_{\sigma \in S_n} f(\sigma_<) &= \sum_{\sigma \in S_n} f((\overline \sigma)_<)
        &\text{pois $\sigma \mapsto \overline \sigma$ é uma bijeção} \\
        &= \sum_{\sigma \in S_n} f(\overline{\sigma_>})
        &\text{pela equação~\ref{eq:conjugate-permutation}} \\
        &= \sum_{\sigma \in S_n} f(\sigma_>)
        &\text{pois $\sigma \mapsto \overline \sigma$ é uma bijeção.}
    \end{align*}
\end{proof}

Como $\sigma_<$ será uma permutação,
o somatório do lado direito da equação~\ref{eq:lower-permutation-to-upper}
pode ser reescrito sem usar o subscrito ${}_<$.
O seguinte lema diz como.

\begin{lemma}
    Se $f: \mathcal S \to \mathbb N$ é uma função qualquer,
    então
    \begin{equation}
        \sum_{\sigma \in S_n} f(\sigma_<)
            = \sum_{k = 0}^{n-1} \frac{(n-1)!}{k!} \sum_{\sigma \in S_k} f(\sigma).
        \label{eq:sum-partitions}
    \end{equation}
\end{lemma}

\begin{proof}
    Primeiro,
    observe que toda permutação $\tau \in S_k$, para $k < n$,
    é $\sigma_<$ para algum $\sigma \in S_n$
    (basta escolher, por exemplo,
    a sequência $(k+1, \tau(1), \dots, \tau(k), k+2, k+3, \dots, n)$);
    portanto,
    a forma do somatório esquerda da equação~\ref{eq:sum-partitions} está certa.
    Falta verificar a constante $(n-1)!/k!$ de cada termo.

    Esta constante refere-se a quantas permutações $\sigma \in S_n$
    satisfazem $\sigma_< = \tau$ para alguma permutação $\tau \in S_k$ dada.
    Fixe, portanto, $\tau \in S_k$.
    Para que $\sigma_< = \tau$,
    primeiro, precisamos ter $\sigma(1) = k+1$,
    pois existem exatamente $k$ elementos em $\sigma$ que são menores que $\sigma(1)$
    --- são esses que virarão os elementos de $\tau$.
    Existem $n-1$ posições que não foram fixadas em $\sigma$;
    dessas, $k$ serão os números $\tau(1), \dots, \tau(k)$,
    em sequência;
    são, portanto, $\binom{n-1}{k}$ diferentes conjuntos de posições
    que terão os elementos de $\tau$.

    Fixadas essas posições,
    precisamos distribuir os $n - k - 1$ números entre $k+1$ e $n$
    nas $n - k - 1$ posições restantes.
    Como a ordem não altera o valor de $\sigma_<$,
    todas as $(n - k - 1)!$ permutações são válidas.
    Assim, o número total de permutações $\sigma \in S_n$
    tal que $\sigma_< = \tau$ é $\binom{n-1}{k} (n-k-1)! = (n-1)!/k!$.
\end{proof}

De posse destes dois lemas,
podemos demonstrar o teorema.

\begin{theorem}
    A altura média de uma árvore de busca binária
    construída a partir de uma permutação de $n$ elementos
    é menor ou igual a $3 \log_2 n$, para $n \geq 1$.
\end{theorem}

(A demonstração construída nesta seção é adaptada do livro de Cormen et al.%
~\cite[p.~300]{CormenLeisersonRivestStein2009}.)

\begin{proof}
    Chame de $H$ a função que retorna a altura da árvore binária de busca
    construída a partir de uma permutação dada;
    isto é, $H(\sigma)$ é a altura da árvore construída a partir de $\sigma$.
    A altura média,
    dentre todas as permutações,
    da árvore de busca,
    é o número $X_n$ definido por
    \begin{equation*}
        X_n = \frac{1}{n!} \sum_{\sigma \in S_n} H(\sigma).
    \end{equation*}
    Iremos trabalhar não com a altura média,
    mas com a ``altura exponencial'' média;
    isto é, com a média do valor de $2^{H(\sigma)}$.
    Defina $Y_n$ por
    \begin{equation}
        Y_n = \frac{1}{n!} \sum_{\sigma \in S_n} 2^{H(\sigma)}.
        \label{eq:y-n-def}
    \end{equation}
    A função $2^x$ é convexa,
    portanto,
    pela desigualdade de Jensen,
    $2^{X_n} \leq Y_n$,
    para $n \geq 1$.
    Assim, basta impor um limite superior a $Y_n$ para provar o teorema.

    Para $n = 0$, o somatório que define $Y_n$ é vazio,
    portanto $Y_0 = 0$.
    Assuma doravante que $n > 0$.

    A função $H$ pode ser definida recursivamente por
    \begin{equation*}
        H(\sigma) = \begin{cases}
            -1, & \text{se $\sigma$ é a permutação vazia.} \\
            1 + \max\{H(\sigma_<), H(\sigma_>)\}, & \text{caso contrário.}
        \end{cases}
    \end{equation*}
    Como $n > 0$, para $\sigma \in S_n$,
    o número $2^{H(\sigma)}$ é igual a $2 \max\{2^{H(\sigma_<)}, 2^{H(\sigma_>)}\}$.
    Tanto $2^{H(\sigma_<)}$ quanto $2^{H(\sigma_>)}$ são positivos,
    assim o máximo entre estes dois números será menor do que sua soma.
    Aplicando estas ideias à equação~\ref{eq:y-n-def},
    temos
    \begin{align*}
        Y_n &= \frac{1}{n!} \sum_{\sigma \in S_n} 2^{H(\sigma)} \\
            &< \frac{2}{n!} \sum_{\sigma \in S_n} 2^{H(\sigma_<)} + 2^{H(\sigma_>)} \\
            &= \frac{4}{n!} \sum_{\sigma \in S_n} 2^{H(\sigma_<)}
        &\text{pela equação~\ref{eq:lower-permutation-to-upper}} \\
        &= \frac{4}{n!} \sum_{k = 0}^{n-1}
            \frac{(n-1)!}{k!} \sum_{\sigma \in S_k} 2^{H(\sigma)}
            &\text{pela equação~\ref{eq:sum-partitions}} \\
        &= \frac{4}{n} \sum_{k = 0}^{n-1}
            \frac{1}{k!} \sum_{\sigma \in S_k} 2^{H(\sigma)} \\
        &= \frac 4 n \sum_{k = 0}^{n-1} Y_k.
    \end{align*}
    Assim, para $n > 0$, $Y_n < \frac 4 n \sum_{k = 0}^{n-1} Y_k$.
    Agora, usaremos esta recorrência para encontrar um limite superior para $Y_n$.

    Defina os números $Z_n$ por $Z_1 = 1$,
    $Z_n = \frac 4 n \sum_{k = 0}^{n-1} Z_k$ para $n > 1$.
    (Os $Z_n$ satisfazem a mesma recorrência que os $Y_n$,
    mas com igualdade em vez de ``menor que'';
    isso facilitará sua manipulação.)
    Observe que $Y_n \leq Z_n$.
    Manipulando a recorrência, obtemos, para $n \geq 1$,
    \begin{align*}
        n Z_n &= 4 \sum_{k = 0}^{n-1} Z_k \\
        (n-1) Z_{n-1} &= 4 \sum_{k = 0}^{n-2} Z_k \\
        n Z_n - (n-1) Z_{n-1} &= 4 Z_{n-1} \\
        n Z_n &= (n+3) Z_{n-1} \\
        Z_n &= \frac{n+3}{n} Z_{n-1} \\
            &= \frac{n+3}{n} * \frac{n+2}{n-1} Z_{n-2} \\
            &= \frac{n+3}{n} * \frac{n+2}{n-1} * \dots * \frac 6 3 * \frac 5 2 Z_1 \\
            &= \frac{(n+3)(n+2)(n+1)}{4*3*2} \\
            &= \frac{n^3 + 6n^2 + 11n + 6}{24} \\
            &\leq \frac {n^3 + 6n^3 + 11n^3 + 6n^3}{24} = n^3.
    \end{align*}
    Encadeando as desigualdades, temos $2^{X_n} \leq Y_n \leq Z_n \leq n_3$.
    Portanto, concluímos que $X_n \leq \log_2(n^3) = 3 \log_2 n$.
\end{proof}

É interessante notar que,
como a altura mínima de uma árvore de busca binária é $\rfloor \log_2 n \rfloor$,
embora pareça que tenhamos ``jogado bastante coisa fora'' na demonstração
(em particular, na última sequência de desigualdades, em que só pegamos os dois extremos),
ainda conseguimos provar que a altura média é menor ou igual a $3 \log_2 n$;
portanto,
a altura média é pouco menos de três vezes a altura mínima.
Mesmo assim,
esta estimativa é surpreendentemente precisa:
$3 \log_2 n = (3/\ln 2) \ln n \approx 4.328 \ln n$.
O valor assintótico de $X_n$ é $c \ln n$,
em que $c \approx 4.31107$ é a única solução em $(2, \infty)$ da equação $c \ln(2e/c) = 1$
\cite[p.~308]{SedgewickFlajolet2013}.
