\subsection{Geração uniforme de árvores binárias}

Como mostrado anteriormente,
embora o procedimento de gerar aleatoriamente uma permutação
e construir uma árvore binária de busca
seja um método efetivo para gerar aleatoriamente árvores binárias,
esse procedimento não gerará todas as árvores de $n$ nós uniformemente.

Uma maneira simples de obter uma distribuição uniforme
é usar os números de Catalão.
Na demonstração da fórmula~\ref{eq:catalan},
mostramos que existem $C_k C_{n-k-1}$ árvores binárias com $n$ nós
tais que a subárvore esquerda possui $k$ nós.
Assim,
poderíamos calcular todos esses valores,
ponderar a geração do número $k$ por eles,
e gerar recursivamente duas árvores binárias
--- uma com $k$ nós e a outra com $n - k - 1$ nós.

Nesta seção,
descreveremos outra maneira de gerar uniformemente uma árvore binária,
que não possui o problema de precisar números ``grandes''
(lembre-se de que os $C_n$ crescem exponencialmente)
ou com possíveis erros de arredondamento
(caso usemos números em ponto flutuante para fazer o ponderamento).

Iremos construir uma bijeção entre árvores binárias e sequências de apostas,
que estão definidas a seguir.

\begin{definition}
    Uma \emph{sequência de apostas} de $2n+1$ bits
    é uma palavra binária com $n$ bits $1$ e $n+1$ bits $0$,
    tal que
    nenhum prefixo próprio possui mais zeros que uns.
\end{definition}

Por exemplo, $11100100100$ é uma sequência de apostas.

O nome desta sequência vem da analogia com um apostador
que começa com uma ficha e faz sucessivas apostas.
Em cada vitória, ele ganha mais uma ficha e marcamos $1$ em sua sequência,
e em cada derrota, ele perde uma ficha e marcamos $0$ em sua sequência.
No instante em que o apostador acumular mais derrotas do que vitórias,
ele terá perdido todas as suas fichas, inclusive a inicial,
portanto sua sequência de apostas encerra-se aqui.
A figura~\ref{fig:apostas}
mostra o gráfico do saldo resultante da sequência de apostas $11100100100$.

\begin{figure}[h]
    \centering
    \begin{tikzpicture}[scale=0.75]
        \datavisualization [school book axes, visualize as line]
        data [separator=\space] {
            x y
            0 0
            1 1
            2 2
            3 3
            4 2
            5 1
            6 2
            7 1
            8 0
            9 1
            10 0
            11 -1
        };
    \end{tikzpicture}
    \caption{
        Gráfico do saldo relativo à sequência de apostas $11100100100$.
        Observe que o único momento em que o saldo do jogador fica negativo
        é no final da sequência
        --- que é quando ele perde a ficha inicial.
    }
    \label{fig:apostas}
\end{figure}

A seguir,
provaremos dois teoremas
que nos darão o número de sequências de apostas com $2n+1$ bits.

\begin{definition}
    Uma \emph{permutação circular}
    de uma cadeia de caracteres $w$
    é uma cadeia da forma $yx$,
    sendo que $w = xy$ e $x \neq \epsilon$.
\end{definition}

Estamos ignorando aqui a decomposição $\epsilon w$
pois ela é funcionalmente idêntica à decomposição $w \epsilon$
e só ``atrapalha'' os enunciados dos teoremas.
Assim,
existem $n$ permutações circulares (não necessariamente distintas)
de uma palavra com $n$ elementos.

\begin{lemma}
    Se $w$ é uma palavra com $n$ uns e $n+1$ zeros,
    então todas as $2n+1$ permutações circulares de $w$ são distintas.
\end{lemma}

\begin{proof}
    Suponha que haja duas decomposições $w = xy = x'y'$
    tais que $|x| < |x'|$ e $yx = y'x'$.
    Como $xy = x'y'$
    existe alguma cadeia $z \neq \epsilon$
    tal que $x' = xz$
    (pois $x$ é um prefixo próprio de $x'$).
    Assim, temos
    \begin{equation*}
        x'y' = xzy' = xy,
    \end{equation*}
    o que mostra que $y = zy'$.
    Agora,
    como $yx = y'x'$,
    temos
    \begin{equation*}
        zy'x = y'xz.
    \end{equation*}
    Escolhendo $u = y'x$, temos $uz = zu$
    para $u, z \neq \epsilon$.
    É um conhecido teorema da análise combinatória em palavras
    que essa situação só pode ocorrer se
    existir alguma cadeia $v$ e inteiros positivos $k, l$ tais que
    $u = v^k$ e $z = v^l$ \cite[p.~32]{Shallit2008}.
    Isto é, tanto $u$ quanto $z$ são potências da mesma palavra.
    Assim,
    se a diferença entre o número de zeros e uns nesta palavra $v$ for $j$,
    a diferença entre o número de zeros e uns em $w$ será $j(k + l)$.
    Como $k + l \geq 2$,
    esse número nunca poderá ser $1$,
    contradizendo o fato de $w$ ter exatamente $n$ uns e $n+1$ zeros.

    Portanto, todas as $2n+1$ permutações circulares de $w$ são distintas.
\end{proof}

Concluímos assim que,
se o conjunto das $\binom{2n+1}{n}$ palavras binárias
for particionado através de permutações circulares
(isto é, duas cadeias são equivalentes se uma é uma permutação circular da outra),
cada classe de equivalência terá exatamente $2n+1$ palavras.
Assim,
serão $\binom{2n+1}{n}/(2n+1) = \frac{1}{n+1} \binom{2n}{n} = C_n$
classes de equivalência
--- exatamente o mesmo número de árvores binárias com $n$ nós.
Assim,
sabemos que há uma bijeção entre árvores binárias e essas classes de equivalência.
