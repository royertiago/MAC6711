\subsection{Geração uniforme de árvores binárias}

Como mostrado anteriormente,
embora o procedimento de gerar aleatoriamente uma permutação
e construir uma árvore binária de busca
seja um método efetivo para gerar aleatoriamente árvores binárias,
esse procedimento não gerará todas as árvores de $n$ nós uniformemente.

Uma maneira simples de obter uma distribuição uniforme
é usar os números de Catalão.
Na derivação da fórmula~(\ref{eq:catalan}),
mostramos que existem $C_k C_{n-k-1}$ árvores binárias com $n$ nós
tais que a subárvore esquerda possui $k$ nós.
Assim,
poderíamos calcular todos esses valores,
ponderar a geração do número $k$ por eles,
e gerar recursivamente duas árvores binárias
--- uma com $k$ nós e a outra com $n - k - 1$ nós.

Nesta seção,
descreveremos outra maneira de gerar uniformemente uma árvore binária,
que não possui o problema de precisar de números grandes
(pois os $C_n$ crescem exponencialmente)
ou com possíveis erros de arredondamento
(caso usemos números em ponto flutuante para fazer o ponderamento).

Construiremos uma bijeção entre árvores binárias e sequências de apostas,
que estão definidas a seguir.

\begin{definition}
    Uma \emph{sequência de apostas}\footnote{
        Este termo é uma tradução livre e aproximada de \emph{gambler's ruin sequences},
        o termo usado por Sedgewick e Flajolet~\cite[p.~268]{SedgewickFlajolet2013}.
    }
    de $2n+1$ bits
    é uma palavra binária com $n$ bits $1$ e $n+1$ bits $0$,
    tal que
    nenhum prefixo próprio possui mais zeros que uns.
\end{definition}

Por exemplo, $11100100100$ é uma sequência de apostas.

O nome desta sequência vem da analogia com um apostador
que começa com uma ficha e faz sucessivas apostas.
Em cada vitória, ele ganha mais uma ficha e marcamos $1$ em sua sequência,
e em cada derrota, ele perde uma ficha e marcamos $0$ em sua sequência.
No instante em que o apostador acumular mais derrotas do que vitórias,
ele terá perdido todas as suas fichas, inclusive a inicial,
portanto sua sequência de apostas encerra-se aqui.
A Figura~\ref{fig:apostas}
mostra o gráfico do saldo resultante da sequência de apostas $11100100100$.

\begin{figure}[h]
    \centering
    \begin{tikzpicture}[scale=0.75]
        \datavisualization [school book axes, visualize as line]
        data [separator=\space] {
            x y
            0 0
            1 1
            2 2
            3 3
            4 2
            5 1
            6 2
            7 1
            8 0
            9 1
            10 0
            11 -1
        };
    \end{tikzpicture}
    \caption{
        Gráfico do saldo relativo à sequência de apostas $11100100100$.
        Observe que o único momento em que o saldo do jogador fica negativo
        é no final da sequência
        --- que é quando ele perde a ficha inicial.
    }
    \label{fig:apostas}
\end{figure}

A seguir,
provaremos dois teoremas
que nos darão o número de sequências de apostas com $2n+1$ bits.

\begin{definition}
    Uma \emph{permutação circular}
    de uma cadeia de caracteres $w$
    é uma cadeia da forma $yx$,
    sendo que $w = xy$ e $x \neq \epsilon$.
\end{definition}

Estamos ignorando aqui a decomposição $\epsilon w$
pois ela se comporta da mesma maneira que a decomposição $w \epsilon$
e só ``atrapalha'' os enunciados dos teoremas.
Assim,
existem $n$ permutações circulares (não necessariamente distintas)
de uma palavra com $n$ elementos.

\begin{lemma}
    Se $w$ é uma palavra com $n$ uns e $n+1$ zeros,
    então todas as $2n+1$ permutações circulares de $w$ são distintas.
\end{lemma}

\begin{proof}
    Suponha que haja duas decomposições $w = xy = x'y'$
    tais que $|x| < |x'|$ e $yx = y'x'$.
    Como $xy = x'y'$
    existe alguma cadeia $z \neq \epsilon$
    tal que $x' = xz$
    (pois $x$ é um prefixo próprio de $x'$).
    Assim, temos
    \begin{equation*}
        x'y' = xzy' = xy,
    \end{equation*}
    o que mostra que $y = zy'$.
    Agora,
    como $yx = y'x'$,
    temos
    \begin{equation*}
        zy'x = y'xz.
    \end{equation*}
    Escolhendo $u = y'x$, temos $uz = zu$
    para $u, z \neq \epsilon$.
    É um conhecido teorema da análise combinatória em palavras
    que essa situação só pode ocorrer se
    existir alguma cadeia $v$ e inteiros positivos $k, l$ tais que
    $u = v^k$ e $z = v^l$ \cite[p.~32]{Shallit2008}.
    Isto é, tanto $u$ quanto $z$ são potências da mesma palavra.
    Assim,
    se a diferença entre o número de zeros e uns nesta palavra $v$ for $j$,
    a diferença entre o número de zeros e uns em $w$ será $j(k + l)$.
    Como $k + l \geq 2$,
    esse número nunca poderá ser $1$,
    contradizendo o fato de $w$ ter exatamente $n$ uns e $n+1$ zeros.

    Portanto, todas as $2n+1$ permutações circulares de $w$ são distintas.
\end{proof}

Concluímos assim que,
se o conjunto das $\binom{2n+1}{n}$ palavras binárias com $n$ uns e $n+1$ zeros
for particionado através de permutações circulares
(isto é, duas cadeias são equivalentes se uma é uma permutação circular da outra),
cada classe de equivalência terá exatamente $2n+1$ palavras.
Assim,
serão $\binom{2n+1}{n}/(2n+1) = \frac{1}{n+1} \binom{2n}{n} = C_n$
classes de equivalência
--- exatamente o mesmo número de árvores binárias com $n$ nós.
Portanto,
sabemos que há uma bijeção entre árvores binárias e essas classes de equivalência.

Assim,
se pudermos pôr uma determinada palavra $w$ com $n$ uns e $n+1$ zeros
em uma ``forma canônica''
(isto é, escolher um representante em cada classe de equivalência),
teremos uma bijeção entre essas palavras e árvores binárias.
O teorema a seguir mostra que
sequências de apostas são essa forma canônica.

\begin{lemma}
    Seja $w$ uma palavra com $n$ uns e $n+1$ zeros.
    Então existe uma única permutação circular de $w$
    que é uma sequência de apostas.
\end{lemma}

\begin{proof}
    Escreva $w = w_0 w_1 \dots w_{2n}$.
    Defina $\Delta w[i..j]$ como sendo a diferença entre
    o número de uns e o número de zeros
    da subpalavra $w_i w_{i+1} \dots w_j$.
    Assim,
    por exemplo,
    a condição da sequência de apostas
    diz que $\Delta w[0..k]$ será negativo
    se e só se $k = 2n$.
    Pode ser que $w$ não seja uma sequência de apostas;
    neste caso, $\Delta w[0..k]$ será negativo
    para valores menores de $k$.
    Escolha $k$ de maneira a minimizar o valor de $\Delta w[0..k]$
    (isto é, num gráfico com a palavra $w$ semelhante ao gráfico~\ref{fig:apostas},
    $k$ é um ponto de mínimo global).
    Na existência de vários mínimos,
    pegue o menor $k$ possível.
    Mostraremos que $w' = w_{k+1} w_{k+2} \dots w_{2n} w_0 w_1 \dots w_k$
    é a única permutação circular de $w$
    que é uma sequência de apostas.

    Primeiro,
    mostraremos que $w'$ é,
    de fato,
    uma sequência de apostas.
    Chame de $d = \Delta w[0..k] < 0$ o mínimo global no gráfico.
    Os prefixos de $w'$ de tamanho $j \leq 2n-k$ satisfazem
    $\Delta w'[0..j-1] = \Delta w[k+1..j+k] = \Delta w[0..j+k] - d$.
    Como $\Delta w[0..j+k] \geq d$
    (pois $d$ é mínimo global),
    $\Delta w'[0..j-1] \geq 0$,
    portanto prefixos de tamanho menor ou igual a $2n - k$
    satisfazem à exigência.

    Os prefixos de $w'$ de tamanho $j > 2n - k$
    envolvem caracteres do início de $w$,
    portanto, $\Delta w'[0..j-1] = \Delta w[0..j-2n+k-1] + \Delta w[k+1..2n]$.
    Como $\Delta w[0..2n] = -1$,
    $\Delta w[k+1..2n] = -1-d$;
    assim, $\Delta w'[0..j-1] = \Delta w[0..j-2n+k-1] - 1 - d$.
    $k$ foi escolhido como sendo o primeiro mínimo global,
    portanto $\Delta w[0..j-2n+k-1]$ será estritamente maior que $d$
    para $j - 2n + k - 1 < k$, ou $j-1 < 2n$.
    Assim,
    $\Delta w'[0..j-1] \geq 0$
    se $2n - k \leq j-1 < 2n$,
    provando que $w'$ é uma sequência de apostas.

    Por último, para provar unicidade,
    note que toda permutação circular de $w$ é uma permutação circular de $w'$.
    Para qualquer decomposição $w' = xy$ com $x, y \neq \epsilon$,
    temos
    \begin{equation*}
        \Delta y[0..|y|-1] = \Delta w'[|x|..2n] = \Delta w'[0..2n] - \Delta w'[0..|x|-1]
        < 0,
    \end{equation*}
    pois $\Delta w[0..|x|-1]$ é um prefixo próprio de $w'$,
    sendo portanto, não-negativo,
    e $\Delta w'[0..2n] = -1$.
    Assim, a permutação circular $yx$ de $w$
    possuiria um prefixo próprio com mais zeros que uns,
    não constituindo uma sequência de apostas.

    Concluímos que $w'$ é a única permutação circular de $w$
    que é uma sequência de apostas.
\end{proof}

\begin{corollary}
    Existe um procedimento $\proc{Representative}(w)$
    que devolve a permutação circular de $w$ que é uma sequência de apostas
    (isto é, devolve o representante da classe de equivalência de $w$).
\end{corollary}

\begin{proof}
    Basta procurar o $k$ da demonstração do lema
    e devolver a permutação correspondente.
\end{proof}

Portanto, cada classe de equivalência descrita acima
possui exatamente uma sequência de apostas;
assim, as sequências de apostas com $2n+1$ bits
são equinumerosas às árvores binárias com $n$ nós.
Portanto,
falta uma bijeção simples entre sequências de apostas e árvores binárias.
Usaremos uma linguagem livre de contexto para fazer o mapeamento.

\begin{lemma}
    A gramática $G$ definida pelas produções
    \begin{equation*}
        T \to 0 \mid 1TT
    \end{equation*}
    é não-ambígua, gera somente sequências de apostas,
    e nós internos de uma árvore de parsing de $G$
    são árvores binárias.
\end{lemma}

\begin{proof}
    Para provar que $G$ é não-ambígua,
    podemos, por exemplo, verificar que $G$ é $LL(1)$; % Olinto approves
    como gramáticas $LL(1)$ são não-ambíguas~\cite[p.~233]{AhoLamSethiUllman2006},
    $G$ é não-ambígua.

    As outras duas afirmações podem ser demonstradas através de indução.

    Para a primeira,
    observe que $0$ satisfaz às exigências.
    Se $w = 1xy$ em que $x, y \in L(G)$,
    por indução sabemos que $x$ e $y$ são sequências de apostas.
    Nenhum prefixo próprio de $x$ terá mais zeros que uns,
    assim, todos os prefixos menores que $|1x|$ satisfazem às exigências.
    Como $x$ possui um zero a mais, $1x$ terá o mesmo número de zeros e uns,
    sendo um prefixo ``válido'';
    e, como nenhum prefixo próprio de $y$ possui mais zeros que uns,
    podemos concatená-los com $1x$ para concluir que
    nenhum prefixo próprio de $w$ possui mais zeros que uns.

    Para a segunda,
    observe que os nós internos da árvore de parsing
    correspondem à produção $T \to 1TT$.
    Existem exatamente dois $T$'s nesta produção,
    que, por indução, são árvores binárias.
    Assim, os nós internos de uma árvore de parsing para $G$
    constituem uma árvore binária.
\end{proof}

\begin{corollary}
    Existe uma função bijetiva $\proc{Parse}(w)$
    que, dada uma sequência de apostas $w$ de tamanho $2n+1$,
    devolve uma árvore binária de $n$ nós.
\end{corollary}

\begin{proof}
    Basta fazer o parsing de acordo com a gramática $G$ do lema.
    Como a gramática é não-ambígua,
    o parsing é injetor.
    Uma sequência de apostas $w$ de $2n+1$ bits possui $n$ uns,
    e cada bit $1$ produz exatamente um nó interno no parsing de $G$;
    portanto, o parsing de $w$ produzirá uma árvore binária com $n$ nós.

    Como cada árvore binária corresponde a alguma sequência de apostas
    e essas sequências não se repetem,
    de fato todas as sequências de apostas são geradas pela gramática $G$.
    Assim, $\proc{Parse}$ reconhece todas as sequências de apostas,
    sendo, portanto, uma bijeção entre sequências de apostas e árvores binárias.
\end{proof}

Finalmente, costurando tudo,
obtemos o seguinte algoritmo para gerar uniformemente uma árvore binária aleatória.

\begin{codebox}
    \Procname{$\proc{Random-Binary-Tree}(n)$}
    \li $w[0] \gets w[1] \gets \dots \gets w[n] \gets 0$
    \li $w[n+1] \gets w[n+2] \gets \dots \gets w[2n] \gets 1$
    \li $\proc{Shuffle}(w)$
    \li \Return $\proc{Parse}(\proc{Representative}(w))$
\end{codebox}

\begin{theorem}
    O procedimento $\proc{Random-Binary-Tree}$ acima
    gera aleatoriamente uma árvore binária com $n$ nós de maneira uniforme.
\end{theorem}

\begin{proof}
    As três primeiras linhas constroem uma das $\binom{2n+1}{n}$ palavras
    que possuem $n$ uns e $n+1$ zeros,
    de maneira uniforme.
    A extração do representante é uniforme,
    pois cada representante é representado por exatamente $2n+1$ diferentes palavras,
    e o parsing é bijetivo,
    portanto não altera a distribuição uniforme.
\end{proof}
