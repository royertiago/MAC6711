\subsection{Geração uniforme de árvores binárias}

Como mostrado anteriormente,
embora o procedimento de gerar aleatoriamente uma permutação
e construir uma árvore binária de busca
seja um método efetivo para gerar aleatoriamente árvores binárias,
esse procedimento não gerará todas as árvores de $n$ nós uniformemente.

Uma maneira simples de obter uma distribuição uniforme
é usar os números de Catalão.
Na demonstração da fórmula~\ref{eq:catalan},
mostramos que existem $C_k C_{n-k-1}$ árvores binárias com $n$ nós
tais que a subárvore esquerda possui $k$ nós.
Assim,
poderíamos calcular todos esses valores,
ponderar a geração do número $k$ por eles,
e gerar recursivamente duas árvores binárias
--- uma com $k$ nós e a outra com $n - k - 1$ nós.

Nesta seção,
descreveremos outra maneira de gerar uniformemente uma árvore binária,
que não possui o problema de precisar números ``grandes''
(lembre-se de que os $C_n$ crescem exponencialmente)
ou com possíveis erros de arredondamento
(caso usemos números em ponto flutuante para fazer o ponderamento).

Iremos construir uma bijeção entre árvores binárias e sequências de apostas,
que estão definidas a seguir.

\begin{definition}
    Uma \emph{sequência de apostas} de $2n+1$ bits
    é uma string binária com $n$ bits $1$ e $n+1$ bits $0$,
    tal que
    nenhum prefixo próprio possui mais zeros que uns.
\end{definition}

Por exemplo, $11100100100$ é uma sequência de apostas.

O nome desta sequência vem da analogia com um apostador
que começa com uma ficha e faz sucessivas apostas.
Em cada vitória, ele ganha mais uma ficha e marcamos $1$ em sua sequência,
e em cada derrota, ele perde uma ficha e marcamos $0$ em sua sequência.
No instante em que o apostador acumular mais derrotas do que vitórias,
ele terá perdido todas as suas fichas, inclusive a inicial,
portanto sua sequência de apostas encerra-se aqui.
A figura~\ref{fig:apostas}
mostra o gráfico do saldo resultante da sequência de apostas $11100100100$.

\begin{figure}[h]
    \centering
    \begin{tikzpicture}[scale=0.75]
        \datavisualization [school book axes, visualize as line]
        data [separator=\space] {
            x y
            0 0
            1 1
            2 2
            3 3
            4 2
            5 1
            6 2
            7 1
            8 0
            9 1
            10 0
            11 -1
        };
    \end{tikzpicture}
    \caption{
        Gráfico do saldo relativo à sequência de apostas $11100100100$.
        Observe que o único momento em que o saldo do jogador fica negativo
        é no final da sequência
        --- que é quando ele perde a ficha inicial.
    }
    \label{fig:apostas}
\end{figure}
