\subsection{Altura média de uma árvore de busca binária}

De posse destes dois lemas,
podemos demonstrar o teorema.

\begin{theorem}
    A altura média de uma árvore de busca binária
    construída a partir de uma permutação de $n$ elementos
    é menor ou igual a $3 \log_2 n$, para $n \geq 1$.
    \label{thm:average-tree-depth}
\end{theorem}

(A demonstração construída nesta seção é adaptada do livro de Cormen et al.%
~\cite[p.~300]{CormenLeisersonRivestStein2009}.)

\begin{proof}
    Chame de $H$ a função que devolve a altura da árvore binária de busca
    construída a partir de uma permutação dada;
    isto é, $H(\sigma)$ é a altura da árvore construída a partir de $\sigma$.
    A altura média,
    dentre todas as permutações,
    da árvore de busca,
    é o número $X_n$ definido por
    \begin{equation*}
        X_n = \frac{1}{n!} \sum_{\sigma \in S_n} H(\sigma).
    \end{equation*}
    Trabalharemos não com a altura média,
    mas com a ``altura exponencial'' média;
    isto é, com a média do valor de $2^{H(\sigma)}$.
    Defina $Y_n$ por
    \begin{equation}
        Y_n = \frac{1}{n!} \sum_{\sigma \in S_n} 2^{H(\sigma)}.
        \label{eq:y-n-def}
    \end{equation}
    A função $2^x$ é convexa,
    portanto,
    pela desigualdade de Jensen,
    $2^{X_n} \leq Y_n$,
    para $n \geq 1$.
    Assim, basta impor um limite superior a $Y_n$ para provar o teorema.

    Para $n = 0$, o somatório que define $Y_n$ é vazio,
    portanto $Y_0 = 0$.
    Suponha doravante que $n > 0$.

    A função $H$ pode ser definida recursivamente por
    \begin{equation*}
        H(\sigma) = \begin{cases}
            -1, & \text{se $\sigma$ é a permutação vazia.} \\
            1 + \max\{H(\sigma_<), H(\sigma_>)\}, & \text{caso contrário.}
        \end{cases}
    \end{equation*}
    Como $n > 0$, para $\sigma \in S_n$,
    o número $2^{H(\sigma)}$ é igual a $2 \max\{2^{H(\sigma_<)}, 2^{H(\sigma_>)}\}$.
    Tanto $2^{H(\sigma_<)}$ quanto $2^{H(\sigma_>)}$ são positivos,
    assim o máximo entre estes dois números será menor do que sua soma.
    Aplicando estas ideias à equação~\ref{eq:y-n-def},
    temos
    \begin{align*}
        Y_n &= \frac{1}{n!} \sum_{\sigma \in S_n} 2^{H(\sigma)} \\
            &< \frac{2}{n!} \sum_{\sigma \in S_n} 2^{H(\sigma_<)} + 2^{H(\sigma_>)} \\
            &= \frac{4}{n!} \sum_{\sigma \in S_n} 2^{H(\sigma_<)}
        &\text{pela equação~\ref{eq:lower-permutation-to-upper}} \\
        &= \frac{4}{n!} \sum_{k = 0}^{n-1}
            \frac{(n-1)!}{k!} \sum_{\sigma \in S_k} 2^{H(\sigma)}
            &\text{pela equação~\ref{eq:sum-partitions}} \\
        &= \frac{4}{n} \sum_{k = 0}^{n-1}
            \frac{1}{k!} \sum_{\sigma \in S_k} 2^{H(\sigma)} \\
        &= \frac 4 n \sum_{k = 0}^{n-1} Y_k.
    \end{align*}
    Assim, para $n > 0$, $Y_n < \frac 4 n \sum_{k = 0}^{n-1} Y_k$.
    Agora, usaremos esta recorrência para encontrar um limite superior para $Y_n$.

    Defina os números $Z_n$ por $Z_1 = 1$,
    $Z_n = \frac 4 n \sum_{k = 0}^{n-1} Z_k$ para $n > 1$.
    (Os $Z_n$ satisfazem a mesma recorrência que os $Y_n$,
    mas com igualdade em vez de ``menor que'';
    isso facilitará sua manipulação.)
    Observe que $Y_n \leq Z_n$.
    Manipulando a recorrência, obtemos, para $n \geq 1$,
    \begin{align*}
        n Z_n &= 4 \sum_{k = 0}^{n-1} Z_k \\
        (n-1) Z_{n-1} &= 4 \sum_{k = 0}^{n-2} Z_k \\
        n Z_n - (n-1) Z_{n-1} &= 4 Z_{n-1} \\
        n Z_n &= (n+3) Z_{n-1} \\
        Z_n &= \frac{n+3}{n} Z_{n-1} \\
            &= \frac{n+3}{n} \cdot \frac{n+2}{n-1} Z_{n-2} \\
            &= \frac{n+3}{n} \cdot \frac{n+2}{n-1} \dots \frac 6 3 \cdot \frac 5 2 Z_1 \\
            &= \frac{(n+3)(n+2)(n+1)}{4 \cdot 3 \cdot 2} \\
            &= \frac{n^3 + 6n^2 + 11n + 6}{24} \\
            &\leq \frac {n^3 + 6n^3 + 11n^3 + 6n^3}{24} = n^3.
    \end{align*}
    Encadeando as desigualdades, temos $2^{X_n} \leq Y_n \leq Z_n \leq n^3$.
    Portanto, concluímos que $X_n \leq \log_2(n^3) = 3 \log_2 n$.
\end{proof}

É interessante notar que,
como a altura mínima de uma árvore de busca binária é $\lfloor \log_2 n \rfloor$,
embora pareça que tenhamos ``jogado bastante coisa fora'' na demonstração
(em particular, na última sequência de desigualdades, em que só pegamos os dois extremos),
ainda conseguimos provar que a altura média é menor ou igual a $3 \log_2 n$;
portanto,
a altura média é pouco menos de três vezes a altura mínima.
Mesmo assim,
esta estimativa é surpreendentemente precisa:
$3 \log_2 n = (3/\ln 2) \ln n \approx 4.328 \ln n$.
O valor assintótico de $X_n$ é $c \ln n$,
em que $c \approx 4.31107$ é a única solução em $(2, \infty)$ da equação $c \ln(2e/c) = 1$
\cite[p.~308]{SedgewickFlajolet2013}.
