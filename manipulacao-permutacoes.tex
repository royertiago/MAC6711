\subsection{Manipulação de permutações}

Na construção de uma árvore binária de busca
a partir de uma permutação de $(1, \dots, n)$,
após fixarmos o primeiro número da permutação como a raiz da árvore,
a adição dos próximos elementos colocará
todos os números menores que a raiz na subárvore esquerda
e todos os números maiores que a raiz na subárvore direita.
Na análise, portanto,
precisaremos particionar uma permutação nestes dois subconjuntos.
Esta seção define uma notação para esta operação
e prova dois lemas que serão usados para facilitar sua manipulação.

Por exemplo, ao construirmos uma árvore
a partir da permutação $\sigma = (4, 5, 3, 6, 1, 2)$,
$\sigma(1) = 4$ será a raiz,
a subárvore esquerda será a árvore construída a partir da lista $(3, 1, 2)$,
e a subárvore direita, da lista $(5, 6)$.
Como o valor dos nós da árvore não é relevante (apenas seu valor relativo é),
podemos supor que a subárvore direita foi gerada pela lista $(1, 2)$,
que é a mesma lista,
mas ``normalizada'' para o intervalo $[1, 2]$.
Chamaremos a lista $(3, 1, 2)$,
a parte que vai para a esquerda, de $\sigma_<$
(é a parte de $\sigma$ que é menor que $\sigma(1)$, a raiz),
e a lista $(1, 2)$,
a que foi gerada a partir da lista $(5, 6)$, de $\sigma_>$.

A definição a seguir formaliza estes nomes.

\begin{definition}
    Seja $\sigma \in S_n$ uma permutação.
    Defina $\sigma_<$ e $\sigma_>$
    como sendo a sequência dos elementos menores e maiores que $\sigma(1)$
    (o primeiro elemento)
    respectivamente,
    normalizados para os intervalo $\{1, \sigma(1) - 1\}$ e $\{1, n - \sigma(1)\}$,
    respectivamente.
    Isto é,
    se $\{a_i\}_{i = 1}^{\sigma(1)-1}$ é a sequência crescente
    dos índices satisfazendo $\sigma(a_i) < \sigma(1)$,
    então $\sigma_< \in S_{\sigma(1) - 1}$
    é a permutação que satisfaz $\sigma_<(i) = \sigma(a_i)$ para todo $i$;
    e se $\{b_i\}_{i = 1}^{n - \sigma(1)}$ é a sequência crescente
    dos índices satisfazendo $\sigma(b_i) > \sigma(1)$,
    então $\sigma_> \in S_{n - \sigma(1)}$
    é a permutação que satisfaz $\sigma_>(i) = \sigma(b_i) - \sigma(1)$ para todo $i$.
\end{definition}

A normalização que $\sigma_>$ sofre garante que
tanto $\sigma_<$ quanto $\sigma_>$ sejam permutações.
Como usaremos ambas,
é importante sabermos como converter somatórios sobre um tipo no outro.

\begin{lemma}
    Se $f: \mathcal S \to \mathbb N$ é uma função qualquer,
    então
    \begin{equation}
        \sum_{\sigma \in S_n} f(\sigma_<) = \sum_{\sigma \in S_n} f(\sigma_>).
        \label{eq:lower-permutation-to-upper}
    \end{equation}
\end{lemma}

\begin{proof}
    Se $\sigma \in S_n$,
    defina $\overline \sigma \in S_n$
    (o ``conjugado'' de $\sigma$)
    por $\overline \sigma(i) = n + 1 - \sigma(i)$.
    Esta operação é bijetora
    e inverte a relação de ordem entre os elementos de $\sigma$.
    Dessa forma, os elementos maiores que $\sigma(1)$, em $\sigma$,
    se tornarão menores que $\overline \sigma(1)$, em $\overline \sigma$.
    Portanto,
    os elementos de $(\overline \sigma)_<$ são,
    exatamente,
    os conjugados dos elementos de $\sigma_>$;
    portanto, temos
    \begin{equation}
        (\overline \sigma)_< = \overline{\sigma_>}.
        \label{eq:conjugate-permutation}
    \end{equation}
    Portanto,
    \begin{align*}
        \sum_{\sigma \in S_n} f(\sigma_<) &= \sum_{\sigma \in S_n} f((\overline \sigma)_<)
        &\text{pois $\sigma \mapsto \overline \sigma$ é uma bijeção} \\
        &= \sum_{\sigma \in S_n} f(\overline{\sigma_>})
        &\text{pela equação~\ref{eq:conjugate-permutation}} \\
        &= \sum_{\sigma \in S_n} f(\sigma_>)
        &\text{pois $\sigma \mapsto \overline \sigma$ é uma bijeção.}
    \end{align*}
\end{proof}

Como $\sigma_<$ será uma permutação,
o somatório do lado direito da equação~\ref{eq:lower-permutation-to-upper}
pode ser reescrito sem usar o subscrito ${}_<$.
O seguinte lema diz como.

\begin{lemma}
    Se $f: \mathcal S \to \mathbb N$ é uma função qualquer,
    então
    \begin{equation}
        \sum_{\sigma \in S_n} f(\sigma_<)
            = \sum_{k = 0}^{n-1} \frac{(n-1)!}{k!} \sum_{\sigma \in S_k} f(\sigma).
        \label{eq:sum-partitions}
    \end{equation}
\end{lemma}

\begin{proof}
    Primeiro,
    observe que toda permutação $\tau \in S_k$, para $k < n$,
    é $\sigma_<$ para algum $\sigma \in S_n$
    (basta escolher, por exemplo,
    a sequência $(k+1, \tau(1), \dots, \tau(k), k+2, k+3, \dots, n)$);
    portanto,
    a forma do somatório do lado direito da equação~\ref{eq:sum-partitions} está certa.
    Falta verificar a constante $(n-1)!/k!$ de cada termo.

    Esta constante refere-se a quantas permutações $\sigma \in S_n$
    satisfazem $\sigma_< = \tau$ para alguma permutação $\tau \in S_k$ dada.
    Fixe, portanto, $\tau \in S_k$.
    Para que $\sigma_< = \tau$,
    primeiro, precisamos ter $\sigma(1) = k+1$,
    pois existem exatamente $k$ elementos em $\sigma$ que são menores que $\sigma(1)$
    --- são esses que virarão os elementos de $\tau$.
    Existem $n-1$ posições que não foram fixadas em $\sigma$;
    dessas, $k$ serão os números $\tau(1), \dots, \tau(k)$,
    em sequência;
    são, portanto, $\binom{n-1}{k}$ diferentes conjuntos de posições
    que terão os elementos de $\tau$.

    Fixadas essas posições,
    precisamos distribuir os $n - k - 1$ números entre $k+1$ e $n$
    nas $n - k - 1$ posições restantes.
    Como a ordem não altera o valor de $\sigma_<$,
    todas as $(n - k - 1)!$ permutações são válidas.
    Assim, o número total de permutações $\sigma \in S_n$
    tal que $\sigma_< = \tau$ é $\binom{n-1}{k} (n-k-1)! = (n-1)!/k!$.
\end{proof}
