\subsection{Comprimento total dos caminhos internos}

\begin{definition}
    Dada uma árvore binária $T$ e um nó interno $v$ de $T$,
    o comprimento do caminho interno de $v$
    é a quantidade de vértices no caminho de $v$ até a raiz de $T$,
    excluindo o próprio vértice
    --- isto é, é a quantidade de arestas entre $v$ e a raiz.
    O comprimento total é simplesmente a soma,
    para todos os nós,
    dos comprimentos de seus caminhos internos.
    \cite[p.~272]{SedgewickFlajolet2013}
\end{definition}

Dada uma permutação $\sigma$,
denotaremos o comprimento total dos caminhos internos
da árvore de busca binária construída a partir de $\sigma$ por $\ipl(\sigma)$.

Para construir uma árvore de busca binária a partir de uma permutação,
cada possível caminho entre a raiz e um nó é percorrido exatamente uma vez,
no momento de inserir este nó,
para descobrir sua posição;
portanto,
$\ipl(\sigma)$ é o custo de construção da árvore de busca binária associada a $\sigma$;
portanto,
é interessante possuir uma estimativa do valor médio de $\ipl(\sigma)$
para $\sigma \in S_n$.

Defina $D_n$ por
\begin{equation*}
    D_n = \frac{1}{n!} \sum_{\sigma \in S_n} \ipl(\sigma).
\end{equation*}
Dada uma permutação $\sigma$,
o custo de inserir a raiz é nulo,
pois sabemos de antemão seu lugar final.
Considere as subárvores à esquerda e à direita,
geradas por $\sigma_<$ e $\sigma_>$.
Cada vértice, ao ser inserido na subárvore à esquerda,
precisa de uma comparação com a raiz (para saber em qual subárvore será inserido)
e mais o mesmo número de comparações que teria para ser inserido em $\sigma_<$
--- pois é a mesma árvore.
O mesmo vale para $\sigma_>$;
portanto, $\ipl(\sigma) = n-1 + \ipl(\sigma_<) + \ipl(\sigma_>)$.

Substituindo na equação, temos
\begin{align*}
    D_n &= \frac{1}{n!} \sum_{\sigma \in S_n} n-1 + \ipl(\sigma_<) + \ipl(\sigma_>) \\
        &= n-1 + \frac{2}{n!} \sum_{k=0}^{n-1} \frac{(n-1)!}{k!}
            \sum{\sigma \in S_k} \ipl(\sigma) & \text{
            pela equação~\ref{eq:sum-partitions}
        } \\
        &= n-1 + \frac 2 n \sum_{k=0}^{n-1} D_k.
\end{align*}
Multiplicando ambos os lados da equação por $n$, temos
\begin{align*}
    nD_n &= n(n-1) + 2 \sum_{k=0}^{n-1} D_k \\
         &= 2(n-1) + 2 D_{n-1} + (n-1)(n-2) + 2\sum_{k=0}^{n-1} D_k \\
         &= 2(n-1) + 2 D_{n-1} + (n-1)D_{n-1} \\
         &= 2(n-1) + (n+1) D_{n-1}.
\end{align*}
Agora, dividindo ambos os lados da equação por $n(n+1)$,
obtemos uma recorrência simples para $D_n/(n+1)$.
\begin{align*}
    \frac{D_n}{n+1} &= \frac{2(n-1)}{n(n+1)} + \frac{D_{n-1}}{n} \\
        &= \sum_{k=1}^n \frac{2(k-1)}{k(k+1)} + \frac{D_0}{1} \\
        &= \sum_{k=1}^n \frac{2(k-1)}{k(k+1)} \\
        &= 2\sum_{k=1}^n \frac{k}{k(k+1)} - 2 \sum_{k=1}^n \frac{1}{k(k+1)} \\
        &= 2H_{n+1} - 2 - 2(1 - 1/(n+1)),
\end{align*}
em que $H_n = \sum_{k=1}^n 1/n$ são os números harmônicos.

Finalmente, multiplicando por $n+1$ obtemos a equação
\begin{equation}
    D_n = 2(n+1)H_{n+1} - 4n - 2.
    \label{eq:construction-cost}
\end{equation}
Sabe-se que $H_n \sim \ln n$; portanto, provamos o seguinte.

\begin{proposition}
    O custo de construção médio de uma árvore de busca binária
    a partir de uma permutação de $n$ elementos é $O(n \ln n)$.
    \label{thm:construction-cost}
\end{proposition}
