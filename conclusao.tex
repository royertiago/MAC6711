\section{Considerações Finais}

Conforme observado no texto,
árvores de busca binárias geradas a partir de permutações aleatórias
possuem uma distribuição bastante diferente da uniforme.
A distribuição uniforme é mais complicada de ser gerada
e produz árvores com altura média significativamente maior
do que a distribuição das árvores de busca.

Árvores de busca geradas a partir de permutações aleatórias
possuem altura média pouco inferior a $3\log_2 n$,
ficando próximo do mínimo teórico $\lfloor \log_2 n \rfloor$.
Treaps exploram essa propriedade,
mexendo diretamente na permutação aleatória
para garantir que a profundidade média se mantenha em $3 \log_2 n$.

A implementação de treaps é consideravelmente mais simples do que árvores AVL e RB,
e, como os testes mostraram,
o desempenho é semelhante.
Isso sugere que treaps sejam uma alternativa viável para árvores balanceadas.
