\section{Aplicação: Treaps}

De acordo com o teorema~\ref{thm:average-tree-depth},
uma árvore de busca gerada aleatoriamente
possui altura esperada logarítmica.
Mais específicamente,
uma árvore de $n$ nós
deve ter profundidade pouco inferior a $3 \log_2 n$.
Árvores auto-balanceadas,
como AVL e rubro-negras,
possuem profundidade inferior a $1.441 \log_2 n$ \cite[p.~460]{Knuth1998}
e $2 \log_2 n$ \cite[p.~309]{CormenLeisersonRivestStein2009},
respectivamente.
Portanto,
na média,
árvores binárias de busca geradas aleatoriamente
possuem profundidade similar às árvores auto-balanceadas,
mas sem o overhead de fazer o balanceamento.

Entretanto,
sabe-se muito pouco sobre o comportamento de árvores binárias de busca
quando inserções e remoções são intercaladas%
~\cite[p.~300]{CormenLeisersonRivestStein2009},
portanto é necessário cuidado ao alterar uma árvore gerada aleatoriamente
para que sua ``aleatoriedade'' se mantenha,
o que nos permite analisá-la.

Treaps~\cite{AragonSeidel1989} são uma forma de preservar a aleatoriedade.
A ideia é associar cada elemento a uma prioridade aleatória,
e permutar o vetor de acordo com essa prioridade.
Como a inserção/remoção de elementos
não altera a distribuição de prioridades,
a árvore contituará sendo uma árvore binária de busca aleatória,
preservando a profundidade esperada logarítmica.

Para definir treaps, precisamos do conceito de \emph{heap}.

\begin{definition}
    Uma árvore binária em que todos os nós possuem uma prioridade associada
    é um \emph{heap}
    se todos os nós da árvore tiverem prioridade maior do que seus filhos.
\end{definition}

É importante notar que o conceito alternativo de heap na literatura;
em textos como o livro de Cormen et. al.~\cite[p.~152]{CormenLeisersonRivestStein2009},
um heap, além de satisfazer a exigência acima,
ainda precisa ser uma árvore binária completa ou quase completa
--- isto é, todos os níveis precisam estar completos,
com excessão do último, que pode estar incompleto,
mas precisa ser preenchido da esquerda para a direita.
Esta restrição adicional
faz com que heaps possam ser eficientemente implementadas como um vetor,
tanto acelerando acesso e modificação quanto reduzindo o espaço utilizado.
(Note que todas essas otimizações apenas mexem na constante associada.)
Alguns textos, como o livro de Sedgewick e Flajolet~\cite[p.~362]{SedgewickFlajolet2013},
chamam a condição acima de ``ordem de heap'';
então, o que eles chamam de ``árvore em ordem de heap``,
nós chamaremos simplesmente de ``heap''.

Ao contrário das árvores binárias,
existem tantas heaps quantas permutações sobre $n$ elementos.

\begin{proposition}
    Existem $n!$ diferentes heaps de $n$ elementos
    cujas prioridades são números do conjunto $\{1, \dots, n\}$.
\end{proposition}

\begin{proof}
    Iremos construir uma bijeção entre os dois conjuntos.
    A figura~\ref{fig:heap-permutation} contém um exemplo deste mapeamento.

    \begin{figure}[h]
        \centering
        \begin{tikzpicture}[binary tree layout, nodes={circle, draw}]
            \node {8}
            child { node {4}
                child { node {2} }
                child { node {3} }
            }
            child { node {7}
                child[missing]
                child { node {6}
                    child { node {1} }
                    child { node {5} }
                }
            };
        \end{tikzpicture}
        \caption{Heap correspondente à permutação $(2, 4, 3, 8, 7, 1, 6, 5)$.}
        \label{fig:heap-permutation}
    \end{figure}

    Dado um heap sobre o conjunto $\{1, \dots, n\}$,
    construa uma permutação
    simplesmente listando as prioridades ao percorrer a árvore em ordem simétrica
    (in-order).
    Dois heaps distintos produzirão permutações diferentes,
    portanto, existem no máximo $n!$ heaps distintas.

    Agora, dada uma permutação $\sigma$,
    escolha $k$ tal que $\sigma(k) = n$; este é o maior valor da permutação.
    Construa um heap colocando o elemento $k$ na raíz,
    e então construa a subárvore esquerda recursivamente com a lista
    $(\sigma(1), \dots, \sigma(k-1))$,
    e a subárvore direita com a lista $(\sigma(k+1), \dots, \sigma(n))$.
    Duas permutações diferentes diferem ou na posição do maior valor,
    e nesse caso as subárvores geradas terão tamanhos diferentes,
    ou em alguma das duas partições,
    e neste caso alguma das subárvores será diferente.
    Portanto, existem ao menos $n!$ heaps distintas.

    Concluímos que existem exatamente $n!$ heaps distintas.
\end{proof}
