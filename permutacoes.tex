\subsection{Geração uniforme de permutações}

Como a seção~\ref{sec:contagem} mostra,
a distribuição de árvores binárias é diferente
da distribuição das árvores construidas a partir de permutações.
Para gerar aleatoriamente uma árvore de busca binária,
podes gerar aleatoriamente uma permutação
e construir a árvore de busca binária associada.
Esta seção descreve um algoritmo para gerar aleatóriamente uma permutação.

Assumiremos que existe um procedimento $\proc{random}(n)$
que gera um número aleatório entre $0$ e $n-1$, com distribuição uniforme.
Este problema foi bem estudado; ver, por exemplo, o livro~\cite{Knuth1997}.

O algoritmo a seguir foi extraído de~\cite[p.~357]{SedgewickFlajolet2013}.
Assumiremos que o vetor $V$ está indexado em $0$.

\begin{codebox}
    \Procname{$\proc{Shuffle}(V)$}
    \li \For $i \gets 1$ \To $\id{length}(V) - 1$
    \li \Do
            swap $A[i]$ and $A[\proc{random}(i+1)]$
    \End
\end{codebox}

\begin{proposition}
    O algoritmo \proc{Shuffle},
    ao embaralhar um vetor de $n$ elementos,
    produz cada uma das $n!$ permutações
    com igual probabilidade.
\end{proposition}

\begin{proof}
    Mostraremos que, ao final da $i$-ésima iteração,
    os elementos $V[0], \dots, V[i]$
    estarão uniformemente embaralhados.
    Observe que essa propriedade é verdadeira antes de o laço executar,
    pois existe uma única permutação para 1 elemento.

    No início da $i$-ésima iteração,
    os elementos $V[0], \dots, V[i-1]$
    estão uniformemente embaralhados.
    Seja $k \in [0, i]$ o inteiro gerado pela função $\proc{random}$.
    O elemento $V[i]$ será inserido na posição $k$ do vetor,
    portanto,
    podemos particionar as permutações dos elementos $V[0], \dots, V[i]$
    em $i+1$ conjuntos,
    um para cada possível valor de $k$.

    Após selecionado $k$,
    o algoritmo efetivamente calcula uma bijeção entre
    as permutações dos elementos $V[0], \dots, V[i-1]$,
    e as permutações dos elementos $V[0], \dots, V[i]$ tais que 
    $V[k]$ é o elemento que originalmente estava na posição $i$ do vetor.
    Como é uma bijeção,
    a distribuição dos elementos não é alterada;
    assim, o algoritmo escolhe uniformemente uma permutação
    deste conjunto de partições.
    E, como $k$ foi gerado uniformemente,
    cada uma das partições (todas de tamanho $i!$) é equiprovável;
    assim, ao final da $i$-ésima iteração,
    cada uma das $(i+1)!$ permutações dos elementos $V[0], \dots, V[i]$
    é equiprovável.
\end{proof}
