\subsection{Geração uniforme de permutações}

Como a seção~\ref{sec:contagem} mostra,
a distribuição de árvores binárias é diferente
da distribuição das árvores construídas a partir de permutações.
Para gerar aleatoriamente uma árvore de busca binária,
podemos gerar aleatoriamente uma permutação
e construir a árvore de busca binária associada.
Esta seção descreve dois algoritmos para permutar aleatoriamente um vetor.
(Assim, uma permutação aleatória pode ser gerada, por exemplo,
permutando o vetor $(1, \dots, n)$.)
Assumiremos que os vetores estão indexados em $1$.

Assumiremos que existe um procedimento $\proc{random}(n)$
que gera um número aleatório entre $1$ e $n$, com distribuição uniforme.
Este problema foi bem estudado; ver, por exemplo, o livro~\cite{Knuth1997}.

O primeiro algoritmo foi extraído de~\cite[p.125]{CormenLeisersonRivestStein2009}.
A ideia é gerar uma prioridade aleatória para cada elemento do vetor
e ordená-lo de acordo com a prioridade.
Se todas as prioridades forem diferentes,
conseguimos garantir que a permutação resultante será gerada uniformemente.

De acordo com Cormen et al. \cite[p.~125]{CormenLeisersonRivestStein2009},
para um vetor de tamanho $n$,
gerar as prioridades no intervalo $[1, n^3]$ resulta numa probabilidade de $1-1/n$
de todas as prioridades serem diferentes.
Alternativamente, poderíamos localizar os grupos com prioridades iguais
e reembaralhar cada grupo;
isso equivale a adicionar casas decimais aleatórias às prioridades,
tornando-as únicas.

\begin{codebox}
    \Procname{$\proc{Shuffle-by-Sorting}(V)$}
    \li $n \gets \id{length}(V)$
    \li let $P$ be a new vector with size $n$
    \li \For $i \gets 1$ \To $n$
    \li \Do
            $P[i] \gets \proc{random}(n^3)$
    \End
    \li Sort $V$ using $P$ as the keys
\end{codebox}

Assumindo que todas as prioridades são diferentes, podemos provar o seguinte teorema.

\begin{proposition}
    O algoritmo \proc{Shuffle-by-Sorting},
    ao embaralhar um vetor de $n$ elementos,
    produz cada uma das $n!$ permutações
    com igual probabilidade.
    \label{thm:shuffle-by-sorting}
\end{proposition}

\begin{proof}
    Primeiro, mostraremos que a probabilidade de este algoritmo
    gerar a permutação identidade é $1/n!$.

    Para isso acontecer,
    o elemento $V[1]$ precisa ter prioridade inferior a todos os outros.
    Como todas as prioridades são diferentes,
    a chance de uma prioridade específica ser a menor é $1/n$;
    portanto, com probabilidade $1/n$ o elemento $V[1]$ terminará na posição $1$.

    Agora, o próximo elemento, $V[2]$,
    precisa ter a menor prioridade dentre a $n-1$ prioridades restantes;
    similarmente, a probabilidade disso ocorrer é $1/(n-1)$.

    Prosseguindo dessa forma,
    concluímos que a probabilidade de o elemento $V[i]$ terminar na posição $i$
    é de $1/(n-i+1)$;
    a probabilidade de gerarmos a permutação identidade
    é o produto destas probabilidades, que é $1/n!$.

    Todas as outras permutações podem ser analisadas similarmente;
    se $\sigma$ é a permutação em questão,
    a probabilidade de que $V[\sigma(1)]$ terminar na posição $1$ é $1/n$,
    e então a probabilidade de $V[\sigma(2)]$ terminar na posição $2$ é $1/(n-1)$,
    e assim por diante.
    A probabilidade de a permutação $\sigma$ ser gerada é $1/n!$.
\end{proof}

Este algoritmo
(em particular sua análise)
será usado na seção~\ref{sec:treaps}.
Entretanto, para o simples propósito de gerar uma permutação aleatória,
existe o algoritmo mais simples a seguir,
extraído de~\cite[p.~357]{SedgewickFlajolet2013}.

\begin{codebox}
    \Procname{$\proc{Shuffle}(V)$}
    \li \For $i \gets 2$ \To $\id{length}(V)$
    \li \Do
            swap $A[i]$ and $A[\proc{random}(i)]$
    \End
\end{codebox}

\begin{proposition}
    O algoritmo \proc{Shuffle},
    ao embaralhar um vetor de $n$ elementos,
    produz cada uma das $n!$ permutações
    com igual probabilidade.
\end{proposition}

\begin{proof}
    Mostraremos que, ao final da $i$-ésima iteração,
    os elementos $V[1], \dots, V[i]$
    estarão uniformemente embaralhados.
    Observe que essa propriedade é verdadeira antes de o laço executar,
    pois existe uma única permutação para 1 elemento.

    No início da $i$-ésima iteração,
    os elementos $V[1], \dots, V[i-1]$
    estão uniformemente embaralhados.
    Seja $k \in [1, i]$ o inteiro gerado pela função $\proc{random}$.
    O elemento $V[i]$ será inserido na posição $k$ do vetor,
    portanto,
    podemos particionar as permutações dos elementos $V[1], \dots, V[i-1]$
    em $i$ conjuntos,
    um para cada possível valor de $k$.

    Após selecionado $k$,
    o algoritmo efetivamente calcula uma bijeção entre
    as permutações dos elementos $V[1], \dots, V[i-1]$,
    e as permutações dos elementos $V[1], \dots, V[i]$ tais que
    $V[k]$ é o elemento que originalmente estava na posição $i$ do vetor.
    Como é uma bijeção,
    a distribuição dos elementos não é alterada;
    assim, o algoritmo escolhe uniformemente uma permutação
    deste conjunto de partições.
    E, como $k$ foi gerado uniformemente,
    cada uma das partições (todas de tamanho $(i-1)!$) é equiprovável;
    assim, ao final da $i$-ésima iteração,
    cada uma das $i!$ permutações dos elementos $V[1], \dots, V[i]$
    é equiprovável.
\end{proof}
