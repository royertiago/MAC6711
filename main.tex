\documentclass{article}
\usepackage[utf8]{inputenc}
\usepackage[T1]{fontenc}

\usepackage[brazil]{babel}

\usepackage{amsmath}
\usepackage{amssymb}
\usepackage{amsthm}

\newtheorem{theorem}{Teorema}[section]
\newtheorem{lemma}[theorem]{Lema}
\newtheorem{corollary}[theorem]{Corolário}
\newtheorem{proposition}[theorem]{Proposição}

\theoremstyle{definition}
\newtheorem{definition}[theorem]{Definição}

\theoremstyle{remark}
\newtheorem*{notation}{Notação}

\usepackage{tikz}
\usetikzlibrary{graphs}
\usetikzlibrary{graphdrawing}
\usegdlibrary{trees}

\usepackage{clrscode3e}

\begin{document}

\title{Árvores de Busca Binária}
\author{Tiago Royer}
\date{28 de abril de 2016}
\maketitle

\begin{abstract}
    Este artigo discute algumas propriedades de árvores de busca binária
    (como, por exemplo, altura média de uma árvore aleatória).
\end{abstract}

\section{Introdução e notação usada}

\begin{definition}
    Uma \emph{árvore binária}
    é ou um único nó externo,
    ou um nó interno,
    que é uma estrutura que contém duas outras árvores
    (a subárvore esquerda e a subárvore direita)
    \cite[p.~257]{SedgewickFlajolet2013}.
\end{definition}

Árvores binárias são o mais simples tipo de árvore
estudados pela Ciência da Computação.
Nas aplicações,
tipicamente ``penduramos'' alguma informação nos nós internos da árvore
e usamos a estrutura da árvore para impor uma relação entre essas informações.
Os nós externos costumam não carregar informação:
eles são usados apenas para simplificar a definição de ``árvore''
--- por exemplo,
em vez de dizer que certo nodo ``não possui uma subárvore esquerda'',
nós dizemos que a subárvore esquerda é um nó externo.
Podemos, por exemplo,
implementar o nó externo como \texttt{null}.
Neste artigo,
desprezaremos a existência desses nós externos
e nos referiremos aos nós internos simplesmente por ``nó'' (ou ``nodo'').
Assim, por exemplo,
uma árvore que só contém um único nó externo
será considerada ``vazia'', ou ``inexistente'' se for subárvore.
Também os excluiremos das contagens;
portanto, consideraremos que
uma árvore que é um nó interno cujas duas subárvores são nós externos
tem apenas um nó.

\begin{definition}
    Uma \emph{árvore binária de busca}
    é uma árvore binária em que
    todos os nós estão associados a alguma chave,
    de forma que a chave de um nó
    é maior que todos os nós de sua subárvore esquerda
    e menor que todos os nós de sua subárvore direita
    \cite[p.~282]{SedgewickFlajolet2013}.
\end{definition}

Existe um procedimento padrão para gerar uma árvore binária de busca
a partir de uma lista de números.
A ideia é inserir os números na ordem em que aparecem,
sempre procurando o ``lugar certo'' na árvore para a inserção.
A figura~\ref{fig:bst-construction} ilustra o processo para a lista $(4, 5, 2, 6, 1, 3)$.

\begin{figure}[h]
    \centering
    \begin{tikzpicture}[binary tree layout, nodes={circle, draw}]
        \node {4};
        \node {4}
        child[missing]
        child { node {5} };
        \node {4}
        child { node {2} }
        child { node {5} };
        \node {4}
        child { node {2} }
        child { node {5}
            child[missing]
            child { node {6} }
        };
        \node {4}
        child { node {2}
            child { node {1} }
        }
        child { node {5}
            child[missing]
            child { node {6} }
        };
        \node {4}
        child { node {2}
            child { node {1} }
            child { node {3} }
        }
        child { node {5}
            child[missing]
            child { node {6} }
        };
    \end{tikzpicture}
    \caption{
        Construção de uma árvore binária de busca a partir da lista $(4, 5, 2, 6, 1, 3)$.
    }
    \label{fig:bst-construction}
\end{figure}

Pela definição,
uma árvore binária de busca não contém chaves repetidas.
Além disso,
para os propósitos deste artigo,
o valor real das chaves não importa
--- o que é relevante é a \emph{ordem} entre eles.
Portanto,
podemos assumir que,
numa árvore com $n$ nós,
seus elementos são os números $1, \dots, n$.
Assim,
a lista que der origem a uma árvore binária de busca
será uma permutação.

Neste artigo,
nos concentraremos nas árvores binárias de busca
e seu contraste estatístico com as árvores binárias uniformes.

\begin{notation}
    Como frequentemente interpretaremos a permutação como uma lista,
    identificaremos as permutações pela lista de seus elementos;
    por exemplo, se $\sigma = (1, 3, 2)$,
    então $\sigma$ é a permutação que satisfaz
    $\sigma(1) = 1, \sigma(2) = 3, \sigma(3) = 2$.

    O conjunto de todas as permutações de $n$ elementos será denotado por $S_n$.
    A permutação vazia, sobre $0$ elementos, é válida;
    assim, $|S_0| = 1$.
    O conjunto de todas as permutações, $\bigcup_{i \in \mathbb N} S_i$,
    será denotado por $\mathcal S$.
\end{notation}


\section{Geração aleatória de árvores binárias}
\subsection{Quantidade de árvores binárias}
\label{sec:contagem}

Sempre que falarmos em gerar algum objeto ``aleatoriamente'',
precisamos discutir qual é a distribuição que estamos gerando.
Para gerar uma árvore binária com $n$ elementos,
podemos, por exemplo,
gerar uma permutação aleatória em $S_n$,
e então convertê-la para uma árvore binária de busca.
Entretanto, desta forma,
não geraremos todas as árvores com $n$ nós de maneira uniforme.
Por exemplo,
para $3$ nós,
ambas as permutações $(2, 1, 3)$ e $(2, 3, 1)$ geram a árvore~%
\tikz \filldraw (0, 0) circle (1pt) -- (.2, .2) circle (1pt) -- (.4, 0) circle (1pt);,
enquanto que apenas a permutação $(1, 2, 3)$ gera a árvore~%
\tikz \filldraw (0, 0) circle (1pt) -- (.1, .1) circle (1pt) -- (.2, .2) circle (1pt);.

Portanto,
concluímos que o número de árvores com $n$ nós
é menor do que o número de permutações com $n$ elementos.
A proposição a seguir quantifica essa afirmação.

\begin{proposition}
    Existem $C_n$ diferentes árvores binárias com $n$ nós,
    em que
    \begin{equation}
        C_n = \frac{1}{n + 1} \binom{2n}{n}
        \label{eq:catalan}
    \end{equation}
    são os números de Catalão.
\end{proposition}

(Essa demonstração foi baseada em~\cite[p.~125]{SedgewickFlajolet2013}.)

\begin{proof}
    Montaremos uma relação de recorrência para $C_n$
    e usaremos funções geradoras para obter a fórmula~\ref{eq:catalan}.

    Temos $C_0 = 1$, pois existe uma única árvore sem nós --- a árvore vazia\footnote{
        Se levarmos em consideração os nós externos da árvore,
        esta árvore é composta de um único nó externo.
    }.
    Para $n > 0$,
    podemos separar as árvores pelo tamanho da subárvore esquerda.
    Fixado um tamanho $k$ para a subárvore esquerda,
    existem $C_{k}$ formas de gerar a subárvore esquerda
    e $C_{n-k-1}$ formas de gerar a subárvore direita.
    Como estas formas são independentes,
    existem $C_{k} C_{n-k-1}$ árvores de $n$ nós
    cuja subárvore esquerda possui $k$ nós.
    Somando esses valores, obtemos a recorrência
    \begin{align*}
        C_n &= \sum_{k = 0}^{n-1} C_k C_{n-k-1} \\
            &= \sum_{k = 1}^n C_{k-1} C_{n-k}.
    \end{align*}
    Seja $f$ a função geradora associada aos números de Catalão.
    Temos
    \begin{align*}
        f(x)^2 &= \left(\sum_{k \geq 0} C_k x^k\right)^2 \\
               &= \sum_{t \geq 0} x^t \left( \sum_{k = 0}^t C_k C_{t-k} \right) \\
               &= \sum_{t \geq 0} x^t C_{t+1} \\
               &= \sum_{t \geq 1} x^{t-1} C_t \\
               &= \frac 1 x \left( -1 + \sum_{t \geq 0} C_t x^t \right) \\
               &= \frac{f(x) - 1}{x}. \\
        x^2 f(x)^2 &= xf(x) - x \\
        \left(xf(x) - \frac 1 2\right)^2 &= \frac 1 4 - x.
    \end{align*}
    Se $x$ estiver próximo de zero,
    então $xf(x) - 1/2$ será próximo de $-1/2$,
    portanto negativo.
    Assim, teremos
    \begin{align*}
        \frac 1 2 - xf(x) &= \sqrt{\frac 1 4 - x} \\
        xf(x) &= \frac 1 2 - \sqrt{\frac 1 4 - x} \\
              &= \frac 1 2 (1 - (1 - 4x)^{1/2})
    \end{align*}
    Agora, podemos expandir o termo $(1 - 4x)^{1/2}$ numa série binomial,
    obtendo
    \begin{align*}
        xf(x) &= \frac 1 2 \left( 1 - \sum_{k \geq 0} \binom{1/2}{k} (-4x)^k \right) \\
              &= -\frac 1 2 \sum_{k \geq 1} \binom{1/2}{k} (-4x)^k \\
        f(x) &= 2 \sum_{k \geq 0} \binom{1/2}{k+1} (-4x)^k
    \end{align*}
    Trabalhando o coeficiente $\binom{1/2}{k+1}$, temos
    \begin{align*}
        \binom{1/2}{k+1} &= \frac{
            \frac 1 2 * (\frac 1 2 - 1) * (\frac 1 2 - 2) * \dots * (\frac 1 2 - k)
        }{ (k+1)! } \\
        &= (-1)^k \frac{
            \frac 1 2 * (1 - \frac 1 2) * (2 - \frac 1 2) * \dots * (k - \frac 1 2)
        }{ (k+1)! } \\
        &= (-1)^k \frac{
            1 * (2 - 1) * (4 - 1) * \dots * (2k - 1)
        }{ 2^{k + 1} (k + 1)! } \\
        &= (-1)^k \frac{ 1 * 1 * 3 * 5 * \dots * (2k - 1) }{2^{k+1} (k+1)!}
            \frac{2 * 4 * 6 * \dots * 2k}{2^k k!} \\
        &= (-1)^k \frac{ (2k)! }{2^{2k+1} (k+1)! k!} \\
        &= \frac{(-1)^k}{2 * (k+1) * 4^k} \binom{2k}{k}.
    \end{align*}
    Substituindo de volta na equação, temos
    \begin{align*}
        f(x) &= 2\sum_{k \geq 0}
            \frac{(-1)^k}{2*(k+1) * 4^k} \binom{2k}{k} (-1)^k 4^k x^k \\
            &= \sum_{k \geq 0} \frac 1 {k+1} \binom{2k} k x^k.
    \end{align*}
    Usando, por exemplo,
    a aproximação de Stirling,
    podemos mostrar que
    \begin{align*}
        \binom{2k} k \sim \frac{4^k}{\sqrt{\pi k}};
    \end{align*}
    isto é, o limite da razão entre essas duas quantidades tende a $1$
    quando $k$ tende ao infinito.
    Portanto, a última série de potências converge para todo $x$ com $|x| < 1/4$.
    Assim, se fizermos as contas de trás para frente,
    todas essas séries de potências convergem.
    Concluímos que a derivação está certa,
    provando, assim, o teorema.
\end{proof}

\subsection{Geração uniforme de permutações}

Como a seção~\ref{sec:contagem} mostra,
a distribuição de árvores binárias é diferente
da distribuição das árvores construidas a partir de permutações.
Para gerar aleatoriamente uma árvore de busca binária,
podes gerar aleatoriamente uma permutação
e construir a árvore de busca binária associada.
Esta seção descreve dois algoritmos para permutar aleatoriamente um vetor.
(Então, uma permutação pode ser gerada, por exemplo,
permutando aleatoriamente o vetor $(1, \dots, n)$.)
Assumiremos que os vetores estão indexados em $0$.

Assumiremos que existe um procedimento $\proc{random}(n)$
que gera um número aleatório entre $0$ e $n-1$, com distribuição uniforme.
Este problema foi bem estudado; ver, por exemplo, o livro~\cite{Knuth1997}.

O primeiro algoritmo foi extraído de~\cite[p.125]{CormenLeisersonRivestStein2009}.
A ideia é gerar uma chave aleatória para cada elemento do vetor
e ordená-lo de acordo com a chave.
Se todas as chaves forem diferentes,
conseguimos garantir que a permutação resultante será gerada uniformemente.

De acordo com Cormen et al. \cite[p.~125]{CormenLeisersonRivestStein2009},
para um vetor de tamanho $n$,
gerar as chaves no intervalo $[1, n^3]$ resulta numa probabilidade de $1-1/n$
de todas as chaves serem diferentes.
Alternativamente, poderíamos localizar os grupos com chaves iguais
e reembaralhar cada grupo;
isso equivale a adicionar dígitos decimais aleatórios às chaves,
tornando-as únicas.

\begin{codebox}
    \Procname{$\proc{Shuffle-by-Sorting}(V)$}
    \li $n \gets \id{length}(V)$
    \li let $K$ be a new vector with size $n$
    \li \For $i \gets 0$ \To $n-1$
    \li \Do
            $K[i] \gets \proc{random}(n^3)$
    \End
    \li Sort $V$ using $K$ as the keys
\end{codebox}

Assumindo que todas as chaves são diferentes, podemos provar o seguinte teorema.

\begin{proposition}
    O algoritmo \proc{Shuffle-by-Sorting},
    ao embaralhar um vetor de $n$ elementos,
    produz cada uma das $n!$ permutações
    com igual probabilidade.
\end{proposition}

\begin{proof}
    (Por simplicidade, indexaremos o vetor em $1$ na análise.)

    Primeiro, mostraremos que a probabilidade de este algoritmo
    gerar a permutação identidade é $1/n!$.

    Para isso acontecer,
    o elemento $V[1]$ precisa ter chave inferior a todos os outros.
    Como todas as chaves são diferentes,
    a chance de uma chave específica ser a menor é $1/n$;
    portanto, com probabilidade $1/n$ o elemento $V[1]$ terminará na posição $1$.

    Agora, o próximo elemento, $V[2]$,
    precisa ter a menor chave dentre a $n-1$ chaves restantes;
    similarmente, a probabilidade disso ocorrer é $1/(n-1)$.

    Prosseguindo dessa forma,
    concluímos que a probabilidade de o elemento $V[i]$ terminar na posição $i$
    é de $1/(n-i+1)$;
    a probabilidade de gerarmos a permutação identidade
    é o produto destas probabilidades, que é $1/n!$.

    Todas as outras permutações podem ser analisadas similarmente;
    se $\sigma$ é a permutação em questão,
    queremos saber a probabilidade de que $V[\sigma(1)]$ terminar na posição $1$,
    e então a probabilidade de $V[\sigma(2)]$ terminar na posição $2$,
    e, em geral, a probabilidade de $V[\sigma(i)]$ terminar na posição $i$.
    a probabilidade de a permutação $\sigma$ ser gerada é $1/n!$.
\end{proof}

Este algoritmo
(em particular sua análise)
será usado na análise das treaps.
Entretanto, para o simples propósito de gerar uma permutação aleatória,
existe o algoritmo mais simples a seguir,
extraído de~\cite[p.~357]{SedgewickFlajolet2013}.

\begin{codebox}
    \Procname{$\proc{Shuffle}(V)$}
    \li \For $i \gets 1$ \To $\id{length}(V) - 1$
    \li \Do
            swap $A[i]$ and $A[\proc{random}(i+1)]$
    \End
\end{codebox}

\begin{proposition}
    O algoritmo \proc{Shuffle},
    ao embaralhar um vetor de $n$ elementos,
    produz cada uma das $n!$ permutações
    com igual probabilidade.
\end{proposition}

\begin{proof}
    Mostraremos que, ao final da $i$-ésima iteração,
    os elementos $V[0], \dots, V[i]$
    estarão uniformemente embaralhados.
    Observe que essa propriedade é verdadeira antes de o laço executar,
    pois existe uma única permutação para 1 elemento.

    No início da $i$-ésima iteração,
    os elementos $V[0], \dots, V[i-1]$
    estão uniformemente embaralhados.
    Seja $k \in [0, i]$ o inteiro gerado pela função $\proc{random}$.
    O elemento $V[i]$ será inserido na posição $k$ do vetor,
    portanto,
    podemos particionar as permutações dos elementos $V[0], \dots, V[i]$
    em $i+1$ conjuntos,
    um para cada possível valor de $k$.

    Após selecionado $k$,
    o algoritmo efetivamente calcula uma bijeção entre
    as permutações dos elementos $V[0], \dots, V[i-1]$,
    e as permutações dos elementos $V[0], \dots, V[i]$ tais que 
    $V[k]$ é o elemento que originalmente estava na posição $i$ do vetor.
    Como é uma bijeção,
    a distribuição dos elementos não é alterada;
    assim, o algoritmo escolhe uniformemente uma permutação
    deste conjunto de partições.
    E, como $k$ foi gerado uniformemente,
    cada uma das partições (todas de tamanho $i!$) é equiprovável;
    assim, ao final da $i$-ésima iteração,
    cada uma das $(i+1)!$ permutações dos elementos $V[0], \dots, V[i]$
    é equiprovável.
\end{proof}


\section{Propriedades Estatísticas de Árvores Binárias Aleatórias}

\subsection{Altura média de uma árvore de busca binária}

(A demonstração construída nesta seção é adaptada do livro de Cormen et al.%
~\cite[p.~300]{CormenLeisersonRivestStein2009}.)

Na construção de uma árvore binária de busca
a partir de uma permutação de $(1, \dots, n)$,
após adicionamos o primeiro número como a raíz,
a adição dos próximos elementos colocará
todos os números menores que a raíz na subárvore à esquerda
e todos os números maiores que a raíz na subárvore à direita.
Na análise, portanto,
precisaremos particionar uma permutação nestes dois subconjuntos.
Assim,
antes de começarmos a conta,
criaremos uma notação para esta operação
e provaremos dois lemas que serão usados para facilitar sua manipulação.

Por exemplo, se a permutação $\sigma$ corresponde a lista $(4, 5, 3, 6, 1, 2)$,
ao construírmos uma árvore a partir de $\sigma$,
$\sigma(1) = 4$ será a raíz,
a subárvore à esquerda será a árvore construída a partir da lista $(3, 1, 2)$,
e a subárvore à direita, da lista $(5, 6)$.
Como o valor dos nós da árvore não é relevante (apenas seu valor relativo é),
podemos supor que a subárvore à direita foi gerada pela lista $(1, 2)$,
que é a mesma lista,
mas ``normalizada'' para o intervalo $[1, 2]$.
Chamaremos a lista $(3, 1, 2)$,
a parte que vai para a esquerda, de $\sigma_<$
(é a parte de $\sigma$ que é menor que $\sigma(1)$, a raíz),
e a lista $(1, 2)$,
a que foi gerada a partir da lista $(5, 6)$, de $\sigma_>$.

A definição a seguir formaliza estes nomes.

\begin{definition}
    Seja $\sigma \in S_n$ uma permutação.
    Defina $\sigma_<$ e $\sigma_>$
    como sendo a sequência dos elementos menores e maiores que $\sigma(1)$
    (o primeiro elemento)
    respectivamente,
    normalizados para os intervalo $\{1, \sigma(1) - 1\}$ e $\{1, n - \sigma(1)\}$,
    respectivamente.
    Isto é,
    se $\{a_i\}_{i = 1}^{\sigma(1)-1}$ é a sequência crescente
    dos índices satisfazendo $\sigma(a_i) < \sigma(1)$,
    então $\sigma_< \in S_{\sigma(1) - 1}$
    é a permutação que satisfaz $\sigma_<(i) = \sigma(a_i)$ para todo $i$;
    e se $\{b_i\}_{i = 1}^{n - \sigma(1)}$ é a sequência crescente
    dos índices satisfazendo $\sigma(b_i) > \sigma(1)$,
    então $\sigma_> \in S_{n - \sigma(1)}$
    é a permutação que satisfaz $\sigma_>(i) = \sigma(b_i) - \sigma(1)$ para todo $i$.
\end{definition}

A normalização que $\sigma_>$ sofre garante que
tanto $\sigma_<$ quanto $\sigma_>$ sejam permutações.
Como usaremos ambas,
é importante sabermos como converter somatórios sobre um tipo no outro.

\begin{lemma}
    Chame de $\mathcal S$ o conjunto de todas as permutações.
    Se $f: \mathcal S \to \mathbb N$ é uma função qualquer,
    então
    \begin{equation}
        \sum_{\sigma \in S_n} f(\sigma_<) = \sum_{\sigma \in S_n} f(\sigma_>).
        \label{eq:lower-permutation-to-upper}
    \end{equation}
\end{lemma}

\begin{proof}
    Se $\sigma \in S_n$,
    defina $\overline \sigma \in S_n$
    (o ``conjugado'' de $\sigma$)
    por $\overline \sigma(i) = n + 1 - \sigma(i)$.
    Esta operação é bijetora
    e inverte a relação de ordem entre os elementos de $\sigma$.
    Dessa forma, os elementos maiores que $\sigma(1)$, em $\sigma$,
    se tornarão menores que $\overline \sigma(1)$, em $\overline \sigma$.
    Portanto,
    os elementos de $(\overline \sigma)_<$ são,
    exatamente,
    os conjugados dos elementos de $\sigma_>$;
    portanto, temos
    \begin{equation}
        (\overline \sigma)_< = \overline{(\sigma_>)}.
        \label{eq:conjugate-permutation}
    \end{equation}
    Portanto,
    \begin{align*}
        \sum_{\sigma \in S_n} f(\sigma_<) &= \sum_{\sigma \in S_n} f((\overline \sigma)_<)
        &\text{pois $\sigma \mapsto \overline \sigma$ é uma bijeção} \\
        &= \sum_{\sigma \in S_n} f(\overline{(\sigma_>)})
        &\text{pela equação~\ref{eq:conjugate-permutation}} \\
        &= \sum_{\sigma \in S_n} f(\sigma_>)
        &\text{pois $\sigma \mapsto \overline \sigma$ é uma bijeção.}
    \end{align*}
\end{proof}


\bibliographystyle{plain}
\bibliography{bib/bibliography}

\end{document}
