\documentclass{article}
\usepackage[utf8]{inputenc}
\usepackage[T1]{fontenc}

\usepackage[brazil]{babel}

\usepackage{booktabs}

\usepackage{amsmath}
\usepackage{amssymb}
\usepackage{amsthm}

\newtheorem{theorem}{Teorema}[section]
\newtheorem{lemma}[theorem]{Lema}
\newtheorem{corollary}[theorem]{Corolário}
\newtheorem{proposition}[theorem]{Proposição}

\theoremstyle{definition}
\newtheorem{definition}[theorem]{Definição}

\theoremstyle{remark}
\newtheorem*{notation}{Notação}

\usepackage{tikz}
\usetikzlibrary{graphs}
\usetikzlibrary{graphdrawing}
\usegdlibrary{trees}
\usetikzlibrary{datavisualization}

\usepackage{clrscode3e}

\DeclareMathOperator{\ipl}{ipl}

\begin{document}

\title{Árvores de Busca Binária}
\author{Tiago Royer}
\date{28 de abril de 2016}
\maketitle

\begin{abstract}
    Este artigo discute sobre a geração aleatória de árvores binárias,
    seus parâmetros estatísticos
    (em particular,
    a altura média de uma árvore de busca binária)
    e uma aplicação destes resultados numa estrutura aleatorizada
    (treaps).
\end{abstract}

\section{Introdução e notação usada}

\begin{definition}
    Uma \emph{árvore binária}
    é ou um único nó externo,
    ou um nó interno,
    que é uma estrutura que contém duas outras árvores
    (a subárvore esquerda e a subárvore direita)
    \cite[p.~257]{SedgewickFlajolet2013}.
\end{definition}

Árvores binárias são o mais simples tipo de árvore
estudados pela Ciência da Computação.
Nas aplicações,
tipicamente ``penduramos'' alguma informação nos nós internos da árvore
e usamos a estrutura da árvore para impor uma relação entre essas informações.
Os nós externos costumam não carregar informação:
eles são usados apenas para simplificar a definição de ``árvore''
--- por exemplo,
em vez de dizer que certo nodo ``não possui uma subárvore esquerda'',
nós dizemos que a subárvore esquerda é um nó externo.
Podemos, por exemplo,
implementar o nó externo como \texttt{null}.
Neste artigo,
desprezaremos a existência desses nós externos
e nos referiremos aos nós internos simplesmente por ``nó'' (ou ``nodo'').
Assim, por exemplo,
uma árvore que só contém um único nó externo
será considerada ``vazia'', ou ``inexistente'' se for subárvore.
Também os excluiremos das contagens;
portanto, consideraremos que
uma árvore que é um nó interno cujas duas subárvores são nós externos
tem apenas um nó.

\begin{definition}
    Uma \emph{árvore binária de busca}
    é uma árvore binária em que
    todos os nós estão associados a alguma chave,
    de forma que a chave de um nó
    é maior que todos os nós de sua subárvore esquerda
    e menor que todos os nós de sua subárvore direita
    \cite[p.~282]{SedgewickFlajolet2013}.
\end{definition}

Existe um procedimento padrão para gerar uma árvore binária de busca
a partir de uma lista de números.
A ideia é inserir os números na ordem em que aparecem,
sempre procurando o ``lugar certo'' na árvore para a inserção.
A figura~\ref{fig:bst-construction} ilustra o processo para a lista $(4, 5, 2, 6, 1, 3)$.

\begin{figure}[h]
    \centering
    \begin{tikzpicture}[binary tree layout, nodes={circle, draw}]
        \node {4};
        \node {4}
        child[missing]
        child { node {5} };
        \node {4}
        child { node {2} }
        child { node {5} };
        \node {4}
        child { node {2} }
        child { node {5}
            child[missing]
            child { node {6} }
        };
        \node {4}
        child { node {2}
            child { node {1} }
        }
        child { node {5}
            child[missing]
            child { node {6} }
        };
        \node {4}
        child { node {2}
            child { node {1} }
            child { node {3} }
        }
        child { node {5}
            child[missing]
            child { node {6} }
        };
    \end{tikzpicture}
    \caption{
        Construção de uma árvore binária de busca a partir da lista $(4, 5, 2, 6, 1, 3)$.
    }
    \label{fig:bst-construction}
\end{figure}

Pela definição,
uma árvore binária de busca não contém chaves repetidas.
Além disso,
para os propósitos deste artigo,
o valor real das chaves não importa
--- o que é relevante é a \emph{ordem} entre eles.
Portanto,
podemos assumir que,
numa árvore com $n$ nós,
seus elementos são os números $1, \dots, n$.
Assim,
a lista que der origem a uma árvore binária de busca
será uma permutação.

Neste artigo,
nos concentraremos nas árvores binárias de busca
e seu contraste estatístico com as árvores binárias uniformes.

\begin{notation}
    Como frequentemente interpretaremos a permutação como uma lista,
    identificaremos as permutações pela lista de seus elementos;
    por exemplo, se $\sigma = (1, 3, 2)$,
    então $\sigma$ é a permutação que satisfaz
    $\sigma(1) = 1, \sigma(2) = 3, \sigma(3) = 2$.

    O conjunto de todas as permutações de $n$ elementos será denotado por $S_n$.
    A permutação vazia, sobre $0$ elementos, é válida;
    assim, $|S_0| = 1$.
    O conjunto de todas as permutações, $\bigcup_{i \in \mathbb N} S_i$,
    será denotado por $\mathcal S$.
\end{notation}


\section{Geração aleatória de árvores binárias}
\subsection{Quantidade de árvores binárias}
\label{sec:contagem}

Sempre que falarmos em gerar algum objeto ``aleatoriamente'',
precisamos discutir qual é a distribuição que estamos gerando.
Para gerar uma árvore binária com $n$ elementos,
podemos, por exemplo,
gerar uma permutação aleatória em $S_n$,
e então convertê-la para uma árvore binária de busca.
Entretanto, desta forma,
não geraremos todas as árvores com $n$ nós de maneira uniforme.
Por exemplo,
para $3$ nós,
ambas as permutações $(2, 1, 3)$ e $(2, 3, 1)$ geram a árvore~%
\tikz \filldraw (0, 0) circle (1pt) -- (.2, .2) circle (1pt) -- (.4, 0) circle (1pt);,
enquanto que apenas a permutação $(1, 2, 3)$ gera a árvore~%
\tikz \filldraw (0, 0) circle (1pt) -- (.1, .1) circle (1pt) -- (.2, .2) circle (1pt);.

Portanto,
concluímos que o número de árvores com $n$ nós
é menor do que o número de permutações com $n$ elementos.
A proposição a seguir quantifica essa afirmação.

\begin{proposition}
    Existem $C_n$ diferentes árvores binárias com $n$ nós,
    em que
    \begin{equation}
        C_n = \frac{1}{n + 1} \binom{2n}{n}
        \label{eq:catalan}
    \end{equation}
    são os números de Catalão.
\end{proposition}

(Essa demonstração foi baseada em~\cite[p.~125]{SedgewickFlajolet2013}.)

\begin{proof}
    Montaremos uma relação de recorrência para $C_n$
    e usaremos funções geradoras para obter a fórmula~\ref{eq:catalan}.

    Temos $C_0 = 1$, pois existe uma única árvore sem nós --- a árvore vazia\footnote{
        Se levarmos em consideração os nós externos da árvore,
        esta árvore é composta de um único nó externo.
    }.
    Para $n > 0$,
    podemos separar as árvores pelo tamanho da subárvore esquerda.
    Fixado um tamanho $k$ para a subárvore esquerda,
    existem $C_{k}$ formas de gerar a subárvore esquerda
    e $C_{n-k-1}$ formas de gerar a subárvore direita.
    Como estas formas são independentes,
    existem $C_{k} C_{n-k-1}$ árvores de $n$ nós
    cuja subárvore esquerda possui $k$ nós.
    Somando esses valores, obtemos a recorrência
    \begin{align*}
        C_n &= \sum_{k = 0}^{n-1} C_k C_{n-k-1} \\
            &= \sum_{k = 1}^n C_{k-1} C_{n-k}.
    \end{align*}
    Seja $f$ a função geradora associada aos números de Catalão.
    Temos
    \begin{align*}
        f(x)^2 &= \left(\sum_{k \geq 0} C_k x^k\right)^2 \\
               &= \sum_{t \geq 0} x^t \left( \sum_{k = 0}^t C_k C_{t-k} \right) \\
               &= \sum_{t \geq 0} x^t C_{t+1} \\
               &= \sum_{t \geq 1} x^{t-1} C_t \\
               &= \frac 1 x \left( -1 + \sum_{t \geq 0} C_t x^t \right) \\
               &= \frac{f(x) - 1}{x}. \\
        x^2 f(x)^2 &= xf(x) - x \\
        \left(xf(x) - \frac 1 2\right)^2 &= \frac 1 4 - x.
    \end{align*}
    Se $x$ estiver próximo de zero,
    então $xf(x) - 1/2$ será próximo de $-1/2$,
    portanto negativo.
    Assim, teremos
    \begin{align*}
        \frac 1 2 - xf(x) &= \sqrt{\frac 1 4 - x} \\
        xf(x) &= \frac 1 2 - \sqrt{\frac 1 4 - x} \\
              &= \frac 1 2 (1 - (1 - 4x)^{1/2})
    \end{align*}
    Agora, podemos expandir o termo $(1 - 4x)^{1/2}$ numa série binomial,
    obtendo
    \begin{align*}
        xf(x) &= \frac 1 2 \left( 1 - \sum_{k \geq 0} \binom{1/2}{k} (-4x)^k \right) \\
              &= -\frac 1 2 \sum_{k \geq 1} \binom{1/2}{k} (-4x)^k \\
        f(x) &= 2 \sum_{k \geq 0} \binom{1/2}{k+1} (-4x)^k
    \end{align*}
    Trabalhando o coeficiente $\binom{1/2}{k+1}$, temos
    \begin{align*}
        \binom{1/2}{k+1} &= \frac{
            \frac 1 2 * (\frac 1 2 - 1) * (\frac 1 2 - 2) * \dots * (\frac 1 2 - k)
        }{ (k+1)! } \\
        &= (-1)^k \frac{
            \frac 1 2 * (1 - \frac 1 2) * (2 - \frac 1 2) * \dots * (k - \frac 1 2)
        }{ (k+1)! } \\
        &= (-1)^k \frac{
            1 * (2 - 1) * (4 - 1) * \dots * (2k - 1)
        }{ 2^{k + 1} (k + 1)! } \\
        &= (-1)^k \frac{ 1 * 1 * 3 * 5 * \dots * (2k - 1) }{2^{k+1} (k+1)!}
            \frac{2 * 4 * 6 * \dots * 2k}{2^k k!} \\
        &= (-1)^k \frac{ (2k)! }{2^{2k+1} (k+1)! k!} \\
        &= \frac{(-1)^k}{2 * (k+1) * 4^k} \binom{2k}{k}.
    \end{align*}
    Substituindo de volta na equação, temos
    \begin{align*}
        f(x) &= 2\sum_{k \geq 0}
            \frac{(-1)^k}{2*(k+1) * 4^k} \binom{2k}{k} (-1)^k 4^k x^k \\
            &= \sum_{k \geq 0} \frac 1 {k+1} \binom{2k} k x^k.
    \end{align*}
    Usando, por exemplo,
    a aproximação de Stirling,
    podemos mostrar que
    \begin{align*}
        \binom{2k} k \sim \frac{4^k}{\sqrt{\pi k}};
    \end{align*}
    isto é, o limite da razão entre essas duas quantidades tende a $1$
    quando $k$ tende ao infinito.
    Portanto, a última série de potências converge para todo $x$ com $|x| < 1/4$.
    Assim, se fizermos as contas de trás para frente,
    todas essas séries de potências convergem.
    Concluímos que a derivação está certa,
    provando, assim, o teorema.
\end{proof}

\subsection{Geração uniforme de permutações}

Como a seção~\ref{sec:contagem} mostra,
a distribuição de árvores binárias é diferente
da distribuição das árvores construidas a partir de permutações.
Para gerar aleatoriamente uma árvore de busca binária,
podes gerar aleatoriamente uma permutação
e construir a árvore de busca binária associada.
Esta seção descreve dois algoritmos para permutar aleatoriamente um vetor.
(Então, uma permutação pode ser gerada, por exemplo,
permutando aleatoriamente o vetor $(1, \dots, n)$.)
Assumiremos que os vetores estão indexados em $0$.

Assumiremos que existe um procedimento $\proc{random}(n)$
que gera um número aleatório entre $0$ e $n-1$, com distribuição uniforme.
Este problema foi bem estudado; ver, por exemplo, o livro~\cite{Knuth1997}.

O primeiro algoritmo foi extraído de~\cite[p.125]{CormenLeisersonRivestStein2009}.
A ideia é gerar uma chave aleatória para cada elemento do vetor
e ordená-lo de acordo com a chave.
Se todas as chaves forem diferentes,
conseguimos garantir que a permutação resultante será gerada uniformemente.

De acordo com Cormen et al. \cite[p.~125]{CormenLeisersonRivestStein2009},
para um vetor de tamanho $n$,
gerar as chaves no intervalo $[1, n^3]$ resulta numa probabilidade de $1-1/n$
de todas as chaves serem diferentes.
Alternativamente, poderíamos localizar os grupos com chaves iguais
e reembaralhar cada grupo;
isso equivale a adicionar dígitos decimais aleatórios às chaves,
tornando-as únicas.

\begin{codebox}
    \Procname{$\proc{Shuffle-by-Sorting}(V)$}
    \li $n \gets \id{length}(V)$
    \li let $K$ be a new vector with size $n$
    \li \For $i \gets 0$ \To $n-1$
    \li \Do
            $K[i] \gets \proc{random}(n^3)$
    \End
    \li Sort $V$ using $K$ as the keys
\end{codebox}

Assumindo que todas as chaves são diferentes, podemos provar o seguinte teorema.

\begin{proposition}
    O algoritmo \proc{Shuffle-by-Sorting},
    ao embaralhar um vetor de $n$ elementos,
    produz cada uma das $n!$ permutações
    com igual probabilidade.
\end{proposition}

\begin{proof}
    (Por simplicidade, indexaremos o vetor em $1$ na análise.)

    Primeiro, mostraremos que a probabilidade de este algoritmo
    gerar a permutação identidade é $1/n!$.

    Para isso acontecer,
    o elemento $V[1]$ precisa ter chave inferior a todos os outros.
    Como todas as chaves são diferentes,
    a chance de uma chave específica ser a menor é $1/n$;
    portanto, com probabilidade $1/n$ o elemento $V[1]$ terminará na posição $1$.

    Agora, o próximo elemento, $V[2]$,
    precisa ter a menor chave dentre a $n-1$ chaves restantes;
    similarmente, a probabilidade disso ocorrer é $1/(n-1)$.

    Prosseguindo dessa forma,
    concluímos que a probabilidade de o elemento $V[i]$ terminar na posição $i$
    é de $1/(n-i+1)$;
    a probabilidade de gerarmos a permutação identidade
    é o produto destas probabilidades, que é $1/n!$.

    Todas as outras permutações podem ser analisadas similarmente;
    se $\sigma$ é a permutação em questão,
    queremos saber a probabilidade de que $V[\sigma(1)]$ terminar na posição $1$,
    e então a probabilidade de $V[\sigma(2)]$ terminar na posição $2$,
    e, em geral, a probabilidade de $V[\sigma(i)]$ terminar na posição $i$.
    a probabilidade de a permutação $\sigma$ ser gerada é $1/n!$.
\end{proof}

Este algoritmo
(em particular sua análise)
será usado na análise das treaps.
Entretanto, para o simples propósito de gerar uma permutação aleatória,
existe o algoritmo mais simples a seguir,
extraído de~\cite[p.~357]{SedgewickFlajolet2013}.

\begin{codebox}
    \Procname{$\proc{Shuffle}(V)$}
    \li \For $i \gets 1$ \To $\id{length}(V) - 1$
    \li \Do
            swap $A[i]$ and $A[\proc{random}(i+1)]$
    \End
\end{codebox}

\begin{proposition}
    O algoritmo \proc{Shuffle},
    ao embaralhar um vetor de $n$ elementos,
    produz cada uma das $n!$ permutações
    com igual probabilidade.
\end{proposition}

\begin{proof}
    Mostraremos que, ao final da $i$-ésima iteração,
    os elementos $V[0], \dots, V[i]$
    estarão uniformemente embaralhados.
    Observe que essa propriedade é verdadeira antes de o laço executar,
    pois existe uma única permutação para 1 elemento.

    No início da $i$-ésima iteração,
    os elementos $V[0], \dots, V[i-1]$
    estão uniformemente embaralhados.
    Seja $k \in [0, i]$ o inteiro gerado pela função $\proc{random}$.
    O elemento $V[i]$ será inserido na posição $k$ do vetor,
    portanto,
    podemos particionar as permutações dos elementos $V[0], \dots, V[i]$
    em $i+1$ conjuntos,
    um para cada possível valor de $k$.

    Após selecionado $k$,
    o algoritmo efetivamente calcula uma bijeção entre
    as permutações dos elementos $V[0], \dots, V[i-1]$,
    e as permutações dos elementos $V[0], \dots, V[i]$ tais que 
    $V[k]$ é o elemento que originalmente estava na posição $i$ do vetor.
    Como é uma bijeção,
    a distribuição dos elementos não é alterada;
    assim, o algoritmo escolhe uniformemente uma permutação
    deste conjunto de partições.
    E, como $k$ foi gerado uniformemente,
    cada uma das partições (todas de tamanho $i!$) é equiprovável;
    assim, ao final da $i$-ésima iteração,
    cada uma das $(i+1)!$ permutações dos elementos $V[0], \dots, V[i]$
    é equiprovável.
\end{proof}

\subsection{Geração uniforme de árvores binárias}

Como mostrado anteriormente,
embora o procedimento de gerar aleatoriamente uma permutação
e construir uma árvore binária de busca
seja um método efetivo para gerar aleatoriamente árvores binárias,
esse procedimento não gerará todas as árvores de $n$ nós uniformemente.

Uma maneira simples de obter uma distribuição uniforme
é usar os números de Catalão.
Na demonstração da fórmula~\ref{eq:catalan},
mostramos que existem $C_k C_{n-k-1}$ árvores binárias com $n$ nós
tais que a subárvore esquerda possui $k$ nós.
Assim,
poderíamos calcular todos esses valores,
ponderar a geração do número $k$ por eles,
e gerar recursivamente duas árvores binárias
--- uma com $k$ nós e a outra com $n - k - 1$ nós.

Nesta seção,
descreveremos outra maneira de gerar uniformemente uma árvore binária,
que não possui o problema de precisar números ``grandes''
(lembre-se de que os $C_n$ crescem exponencialmente)
ou com possíveis erros de arredondamento
(caso usemos números em ponto flutuante para fazer o ponderamento).


\section{Propriedades estatísticas de árvores binárias aleatórias}
Como já visto,
o processo de gerar uma permutação e construir a árvore de busca binária correspondente
não gera uma árvore binária de maneira uniforme;
portanto,
é esperado que alguns parâmetros estatísticos
(como altura média)
sejam diferentes para as duas distribuições.

Esta seção analisa dois parâmetros em particular
(o comprimento interno médio e a altura média)
para árvores de busca binária.
A análise para a distribuição uniforme é consideravelmente mais complexa
e será apenas enunciada.

\subsection{Manipulação de permutações}

Na construção de uma árvore binária de busca
a partir de uma permutação de $(1, \dots, n)$,
após fixarmos o primeiro número da permutação como a raiz da árvore,
a adição dos próximos elementos colocará
todos os números menores que a raiz na subárvore esquerda
e todos os números maiores que a raiz na subárvore direita.
Na análise, portanto,
precisaremos particionar uma permutação nestes dois subconjuntos.
Esta seção define uma notação para esta operação
e prova dois lemas que serão usados para facilitar sua manipulação.

Por exemplo, ao construirmos uma árvore
a partir da permutação $\sigma = (4, 5, 3, 6, 1, 2)$,
$\sigma(1) = 4$ será a raiz,
a subárvore esquerda será a árvore construída a partir da lista $(3, 1, 2)$,
e a subárvore direita, da lista $(5, 6)$.
Como o valor dos nós da árvore não é relevante (apenas seu valor relativo é),
podemos supor que a subárvore direita foi gerada pela lista $(1, 2)$,
que é a mesma lista,
mas ``normalizada'' para o intervalo $[1, 2]$.
Chamaremos a lista $(3, 1, 2)$,
a parte que vai para a esquerda, de $\sigma_<$
(é a parte de $\sigma$ que é menor que $\sigma(1)$, a raiz),
e a lista $(1, 2)$,
a que foi gerada a partir da lista $(5, 6)$, de $\sigma_>$.

A definição a seguir formaliza estes nomes.

\begin{definition}
    Seja $\sigma \in S_n$ uma permutação.
    Defina $\sigma_<$ e $\sigma_>$
    como sendo a sequência dos elementos menores e maiores que $\sigma(1)$
    (o primeiro elemento)
    respectivamente,
    normalizados para os intervalo $\{1, \sigma(1) - 1\}$ e $\{1, n - \sigma(1)\}$,
    respectivamente.
    Isto é,
    se $\{a_i\}_{i = 1}^{\sigma(1)-1}$ é a sequência crescente
    dos índices satisfazendo $\sigma(a_i) < \sigma(1)$,
    então $\sigma_< \in S_{\sigma(1) - 1}$
    é a permutação que satisfaz $\sigma_<(i) = \sigma(a_i)$ para todo $i$;
    e se $\{b_i\}_{i = 1}^{n - \sigma(1)}$ é a sequência crescente
    dos índices satisfazendo $\sigma(b_i) > \sigma(1)$,
    então $\sigma_> \in S_{n - \sigma(1)}$
    é a permutação que satisfaz $\sigma_>(i) = \sigma(b_i) - \sigma(1)$ para todo $i$.
\end{definition}

A normalização que $\sigma_>$ sofre garante que
tanto $\sigma_<$ quanto $\sigma_>$ sejam permutações.
Como usaremos ambas,
é importante sabermos como converter somatórios sobre um tipo no outro.

\begin{lemma}
    Se $f: \mathcal S \to \mathbb N$ é uma função qualquer,
    então
    \begin{equation}
        \sum_{\sigma \in S_n} f(\sigma_<) = \sum_{\sigma \in S_n} f(\sigma_>).
        \label{eq:lower-permutation-to-upper}
    \end{equation}
\end{lemma}

\begin{proof}
    Se $\sigma \in S_n$,
    defina $\overline \sigma \in S_n$
    (o ``conjugado'' de $\sigma$)
    por $\overline \sigma(i) = n + 1 - \sigma(i)$.
    Esta operação é bijetora
    e inverte a relação de ordem entre os elementos de $\sigma$.
    Dessa forma, os elementos maiores que $\sigma(1)$, em $\sigma$,
    se tornarão menores que $\overline \sigma(1)$, em $\overline \sigma$.
    Portanto,
    os elementos de $(\overline \sigma)_<$ são,
    exatamente,
    os conjugados dos elementos de $\sigma_>$;
    portanto, temos
    \begin{equation}
        (\overline \sigma)_< = \overline{\sigma_>}.
        \label{eq:conjugate-permutation}
    \end{equation}
    Portanto,
    \begin{align*}
        \sum_{\sigma \in S_n} f(\sigma_<) &= \sum_{\sigma \in S_n} f((\overline \sigma)_<)
        &\text{pois $\sigma \mapsto \overline \sigma$ é uma bijeção} \\
        &= \sum_{\sigma \in S_n} f(\overline{\sigma_>})
        &\text{pela equação~\ref{eq:conjugate-permutation}} \\
        &= \sum_{\sigma \in S_n} f(\sigma_>)
        &\text{pois $\sigma \mapsto \overline \sigma$ é uma bijeção.}
    \end{align*}
\end{proof}

Como $\sigma_<$ será uma permutação,
o somatório do lado direito da equação~\ref{eq:lower-permutation-to-upper}
pode ser reescrito sem usar o subscrito ${}_<$.
O seguinte lema diz como.

\begin{lemma}
    Se $f: \mathcal S \to \mathbb N$ é uma função qualquer,
    então
    \begin{equation}
        \sum_{\sigma \in S_n} f(\sigma_<)
            = \sum_{k = 0}^{n-1} \frac{(n-1)!}{k!} \sum_{\sigma \in S_k} f(\sigma).
        \label{eq:sum-partitions}
    \end{equation}
\end{lemma}

\begin{proof}
    Primeiro,
    observe que toda permutação $\tau \in S_k$, para $k < n$,
    é $\sigma_<$ para algum $\sigma \in S_n$
    (basta escolher, por exemplo,
    a sequência $(k+1, \tau(1), \dots, \tau(k), k+2, k+3, \dots, n)$);
    portanto,
    a forma do somatório do lado direito da equação~\ref{eq:sum-partitions} está certa.
    Falta verificar a constante $(n-1)!/k!$ de cada termo.

    Esta constante refere-se a quantas permutações $\sigma \in S_n$
    satisfazem $\sigma_< = \tau$ para alguma permutação $\tau \in S_k$ dada.
    Fixe, portanto, $\tau \in S_k$.
    Para que $\sigma_< = \tau$,
    primeiro, precisamos ter $\sigma(1) = k+1$,
    pois existem exatamente $k$ elementos em $\sigma$ que são menores que $\sigma(1)$
    --- são esses que virarão os elementos de $\tau$.
    Existem $n-1$ posições que não foram fixadas em $\sigma$;
    dessas, $k$ serão os números $\tau(1), \dots, \tau(k)$,
    em sequência;
    são, portanto, $\binom{n-1}{k}$ diferentes conjuntos de posições
    que terão os elementos de $\tau$.

    Fixadas essas posições,
    precisamos distribuir os $n - k - 1$ números entre $k+1$ e $n$
    nas $n - k - 1$ posições restantes.
    Como a ordem não altera o valor de $\sigma_<$,
    todas as $(n - k - 1)!$ permutações são válidas.
    Assim, o número total de permutações $\sigma \in S_n$
    tal que $\sigma_< = \tau$ é $\binom{n-1}{k} (n-k-1)! = (n-1)!/k!$.
\end{proof}

\subsection{Comprimento total dos caminhos internos}

\begin{definition}
    Dada uma árvore binária $T$ e um nó interno $v$ de $T$,
    o comprimento do caminho interno de $v$
    é a quantidade de vértices no caminho de $v$ até a raiz de $T$,
    excluindo o próprio vértice
    --- isto é, é a quantidade de arestas entre $v$ e a raiz.
    O comprimento total é simplesmente a soma,
    para todos os nós,
    dos comprimentos de seus caminhos internos.
    \cite[p.~272]{SedgewickFlajolet2013}
\end{definition}

Dada uma permutação $\sigma$,
denotaremos o comprimento total dos caminhos internos
da árvore de busca binária construída a partir de $\sigma$ por $\ipl(\sigma)$.

Para construir uma árvore de busca binária a partir de uma permutação,
cada possível caminho entre a raiz e um nó é percorrido exatamente uma vez,
no momento de inserir este nó,
para descobrir sua posição;
portanto,
$\ipl(\sigma)$ é o custo de construção da árvore de busca binária associada a $\sigma$;
portanto,
é interessante possuir uma estimativa do valor médio de $\ipl(\sigma)$
para $\sigma \in S_n$.

Defina $D_n$ por
\begin{equation*}
    D_n = \frac{1}{n!} \sum_{\sigma \in S_n} \ipl(\sigma).
\end{equation*}
Dada uma permutação $\sigma$,
o custo de inserir a raiz é nulo,
pois sabemos de antemão seu lugar final.
Considere as subárvores à esquerda e à direita,
geradas por $\sigma_<$ e $\sigma_>$.
Cada vértice, ao ser inserido na subárvore à esquerda,
precisa de uma comparação com a raiz (para saber em qual subárvore será inserido)
e mais o mesmo número de comparações que teria para ser inserido em $\sigma_<$
--- pois é a mesma árvore.
O mesmo vale para $\sigma_>$;
portanto, $\ipl(\sigma) = n-1 + \ipl(\sigma_<) + \ipl(\sigma_>)$.

Substituindo na equação, temos
\begin{align*}
    D_n &= \frac{1}{n!} \sum_{\sigma \in S_n} n-1 + \ipl(\sigma_<) + \ipl(\sigma_>) \\
        &= n-1 + \frac{2}{n!} \sum_{k=0}^{n-1} \frac{(n-1)!}{k!}
            \sum{\sigma \in S_k} \ipl(\sigma) & \text{
            pela equação~\ref{eq:sum-partitions}
        } \\
        &= n-1 + \frac 2 n \sum_{k=0}^{n-1} D_k.
\end{align*}
Multiplicando ambos os lados da equação por $n$, temos
\begin{align*}
    nD_n &= n(n-1) + 2 \sum_{k=0}^{n-1} D_k \\
         &= 2(n-1) + 2 D_{n-1} + (n-1)(n-2) + 2\sum_{k=0}^{n-1} D_k \\
         &= 2(n-1) + 2 D_{n-1} + (n-1)D_{n-1} \\
         &= 2(n-1) + (n+1) D_{n-1}.
\end{align*}
Agora, dividindo ambos os lados da equação por $n(n+1)$,
obtemos uma recorrência simples para $D_n/(n+1)$.
\begin{align*}
    \frac{D_n}{n+1} &= \frac{2(n-1)}{n(n+1)} + \frac{D_{n-1}}{n} \\
        &= \sum_{k=1}^n \frac{2(k-1)}{k(k+1)} + \frac{D_0}{1} \\
        &= \sum_{k=1}^n \frac{2(k-1)}{k(k+1)} \\
        &= 2\sum_{k=1}^n \frac{k}{k(k+1)} - 2 \sum_{k=1}^n \frac{1}{k(k+1)} \\
        &= 2H_{n+1} - 2 - 2(1 - 1/(n+1)),
\end{align*}
em que $H_n = \sum_{k=1}^n 1/n$ são os números harmônicos.

Finalmente, multiplicando por $n+1$ obtemos a equação
\begin{equation}
    D_n = 2(n+1)H_{n+1} - 4n - 2.
    \label{eq:construction-cost}
\end{equation}
Sabe-se que $H_n \sim \ln n$; portanto, provamos o seguinte.

\begin{proposition}
    O custo de construção médio de uma árvore de busca binária
    a partir de uma permutação de $n$ elementos é $O(n \ln n)$.
    \label{thm:construction-cost}
\end{proposition}

\subsection{Altura média de uma árvore de busca binária}

Dividindo o lado direito da equação~(\ref{eq:construction-cost}) por $n$,
concluímos que a altura média de um vértice numa árvore de busca binária
é assintoticamente igual a $2 \ln n$.
Como $2 \ln n = 2 \ln 2 \cdot \log_2 n$
e $2 \ln 2 \approx 1.386$,
concluímos que os vértices de árvores de busca binárias
estão aproximadamente $40\%$ mais distantes do que o ótimo, $\log_2 n$.

Entretanto,
isso dá pouca garantia em relação a altura da árvore.
Por exemplo,
uma árvore com $n$ vértices e altura $k$
pode ser constituída de um único ramo com $k$ vértices,
cujo custo de construção soma $k(k-1)/2$,
e os demais vértices completamente balanceados noutro ramo da árvore,
somando pouco menos de $2 (n-k) \log_2 (n-k)$ para o custo de construção.
Escolhendo $k = \sqrt n$,
construímos uma árvore de altura $\sqrt n$,
mas a altura média dos vértices é menor que $4 \log_2 n$.
Assim,
o Teorema~\ref{thm:construction-cost}
garante apenas que altura média de uma árvore não ultrapassará $\sqrt n$.

Mas,
através de uma demonstração diferente,
podemos provar que a altura média é, de fato, logarítmica.

\begin{theorem}
    A altura média de uma árvore de busca binária
    construída a partir de uma permutação de $n$ elementos
    é menor ou igual a $3 \log_2 n$, para $n \geq 1$.
    \label{thm:average-tree-depth}
\end{theorem}

Observe que este teorema implica o Teorema~\ref{thm:construction-cost}.
(A demonstração construída nesta seção é adaptada do livro de Cormen et al.%
~\cite[p.~300]{CormenLeisersonRivestStein2009}.)

\begin{proof}
    Chame de $H$ a função que devolve a altura da árvore binária de busca
    construída a partir de uma permutação dada;
    isto é, $H(\sigma)$ é a altura da árvore construída a partir de $\sigma$.
    A altura média,
    dentre todas as permutações,
    da árvore de busca,
    é o número $X_n$ definido por
    \begin{equation*}
        X_n = \frac{1}{n!} \sum_{\sigma \in S_n} H(\sigma).
    \end{equation*}
    Trabalharemos não com a altura média,
    mas com a ``altura exponencial'' média;
    isto é, com a média do valor de $2^{H(\sigma)}$.
    Defina $Y_n$ por
    \begin{equation}
        Y_n = \frac{1}{n!} \sum_{\sigma \in S_n} 2^{H(\sigma)}.
        \label{eq:y-n-def}
    \end{equation}
    A função $2^x$ é convexa,
    portanto,
    pela desigualdade de Jensen,
    $2^{X_n} \leq Y_n$,
    para $n \geq 1$.
    Assim, basta impor um limite superior a $Y_n$ para provar o teorema.

    Para $n = 0$, o somatório que define $Y_n$ é vazio,
    portanto $Y_0 = 0$.
    Suponha doravante que $n > 0$.

    A função $H$ pode ser definida recursivamente por
    \begin{equation*}
        H(\sigma) = \begin{cases}
            -1, & \text{se $\sigma$ é a permutação vazia.} \\
            1 + \max\{H(\sigma_<), H(\sigma_>)\}, & \text{caso contrário.}
        \end{cases}
    \end{equation*}
    Como $n > 0$, para $\sigma \in S_n$,
    o número $2^{H(\sigma)}$ é igual a $2 \max\{2^{H(\sigma_<)}, 2^{H(\sigma_>)}\}$.
    Tanto $2^{H(\sigma_<)}$ quanto $2^{H(\sigma_>)}$ são positivos,
    assim o máximo entre estes dois números será menor do que sua soma.
    Aplicando estas ideias à equação~\ref{eq:y-n-def},
    temos
    \begin{align*}
        Y_n &= \frac{1}{n!} \sum_{\sigma \in S_n} 2^{H(\sigma)} \\
            &< \frac{2}{n!} \sum_{\sigma \in S_n} 2^{H(\sigma_<)} + 2^{H(\sigma_>)} \\
        &= \frac{4}{n!} \sum_{k = 0}^{n-1}
            \frac{(n-1)!}{k!} \sum_{\sigma \in S_k} 2^{H(\sigma)}
            &\text{pela equação~\ref{eq:sum-partitions}} \\
        &= \frac{4}{n} \sum_{k = 0}^{n-1}
            \frac{1}{k!} \sum_{\sigma \in S_k} 2^{H(\sigma)} \\
        &= \frac 4 n \sum_{k = 0}^{n-1} Y_k.
    \end{align*}
    Assim, para $n > 0$, $Y_n < \frac 4 n \sum_{k = 0}^{n-1} Y_k$.
    Agora, usaremos esta recorrência para encontrar um limite superior para $Y_n$.

    Defina os números $Z_n$ por $Z_1 = 1$,
    $Z_n = \frac 4 n \sum_{k = 0}^{n-1} Z_k$ para $n > 1$.
    (Os $Z_n$ satisfazem a mesma recorrência que os $Y_n$,
    mas com igualdade em vez de ``menor que'';
    isso facilitará sua manipulação.)
    Observe que $Y_n \leq Z_n$.
    Manipulando a recorrência, obtemos, para $n \geq 1$,
    \begin{align*}
        n Z_n &= 4 \sum_{k = 0}^{n-1} Z_k \\
        (n-1) Z_{n-1} &= 4 \sum_{k = 0}^{n-2} Z_k \\
        n Z_n - (n-1) Z_{n-1} &= 4 Z_{n-1} \\
        n Z_n &= (n+3) Z_{n-1} \\
        Z_n &= \frac{n+3}{n} Z_{n-1} \\
            &= \frac{n+3}{n} \cdot \frac{n+2}{n-1} Z_{n-2} \\
            &= \frac{n+3}{n} \cdot \frac{n+2}{n-1} \dots \frac 6 3 \cdot \frac 5 2 Z_1 \\
            &= \frac{(n+3)(n+2)(n+1)}{4 \cdot 3 \cdot 2} \\
            &= \frac{n^3 + 6n^2 + 11n + 6}{24} \\
            &\leq \frac {n^3 + 6n^3 + 11n^3 + 6n^3}{24} = n^3.
    \end{align*}
    Encadeando as desigualdades, temos $2^{X_n} \leq Y_n \leq Z_n \leq n^3$.
    Portanto, concluímos que $X_n \leq \log_2(n^3) = 3 \log_2 n$.
\end{proof}

É interessante notar que,
como a altura mínima de uma árvore de busca binária é $\lfloor \log_2 n \rfloor$,
embora pareça que tenhamos ``jogado bastante coisa fora'' na demonstração
(em particular, na última sequência de desigualdades, em que só pegamos os dois extremos),
ainda conseguimos provar que a altura média é menor ou igual a $3 \log_2 n$;
portanto,
a altura média é pouco menos de três vezes a altura mínima.
Mesmo assim,
esta estimativa é surpreendentemente precisa:
$3 \log_2 n = (3/\ln 2) \ln n \approx 4.328 \ln n$.
O valor assintótico de $X_n$ é $c \ln n$,
em que $c \approx 4.31107$ é a única solução em $(2, \infty)$ da equação $c \ln(2e/c) = 1$
\cite[p.~308]{SedgewickFlajolet2013}.

\subsection{Estatísticas para distribuição uniforme de árvores binárias}

Para árvores binárias uniformemente distribuídas,
a análise é consideravelmente mais complexa,
portanto será apenas enunciada.

O comprimento total dos caminhos internos é
em média,
\begin{equation*}
    \frac{(n+1) 4^n}{\binom{2n}{n}} \sim n \sqrt{\pi n}
\end{equation*}
para uma árvore com $n$ nós~\cite[p.~288]{SedgewickFlajolet2013}.
Portanto,
a profundidade média de um nó é $\sqrt{\pi n}$,
que é significativamente maior do que a profundidade média $2 \ln n$
para árvores de busca binárias.

E, assim como no caso das árvores de busca binárias,
a altura média de uma árvore binária gerada uniformemente
é proporcional a altura média dos nós, $\sqrt{\pi n} + O(1)$
\cite[p.~306]{SedgewickFlajolet2013}.


\section{Aplicação: Treaps}

De acordo com o teorema~\ref{thm:average-tree-depth},
uma árvore de busca gerada aleatoriamente
possui altura esperada logarítmica.
Mais específicamente,
uma árvore de $n$ nós
deve ter profundidade pouco inferior a $3 \log_2 n$.
Árvores auto-balanceadas,
como AVL e rubro-negras,
possuem profundidade inferior a $1.441 \log_2 n$ \cite[p.~460]{Knuth1998}
e $2 \log_2 n$ \cite[p.~309]{CormenLeisersonRivestStein2009},
respectivamente.
Portanto,
na média,
árvores binárias de busca geradas aleatoriamente
possuem profundidade similar às árvores auto-balanceadas,
mas sem o overhead de fazer o balanceamento.

Entretanto,
sabe-se muito pouco sobre o comportamento de árvores binárias de busca
quando inserções e remoções são intercaladas%
~\cite[p.~300]{CormenLeisersonRivestStein2009},
portanto é necessário cuidado ao alterar uma árvore gerada aleatoriamente
para que sua ``aleatoriedade'' se mantenha,
o que nos permite analisá-la.

Treaps~\cite{AragonSeidel1989} são uma forma de preservar a aleatoriedade.
A ideia é associar cada elemento a uma prioridade aleatória,
e permutar o vetor de acordo com essa prioridade.
Como a inserção/remoção de elementos
não altera a distribuição de prioridades,
a árvore contituará sendo uma árvore binária de busca aleatória,
preservando a profundidade esperada logarítmica.

Para definir treaps, precisamos do conceito de \emph{heap}.

\begin{definition}
    Uma árvore binária em que todos os nós possuem uma prioridade associada
    é um \emph{heap}
    se todos os nós da árvore tiverem prioridade maior do que seus filhos.
\end{definition}

É importante notar que o conceito alternativo de heap na literatura;
em textos como o livro de Cormen et. al.~\cite[p.~152]{CormenLeisersonRivestStein2009},
um heap, além de satisfazer a exigência acima,
ainda precisa ser uma árvore binária completa ou quase completa
--- isto é, todos os níveis precisam estar completos,
com excessão do último, que pode estar incompleto,
mas precisa ser preenchido da esquerda para a direita.
Esta restrição adicional
faz com que heaps possam ser eficientemente implementadas como um vetor,
tanto acelerando acesso e modificação quanto reduzindo o espaço utilizado.
(Note que todas essas otimizações apenas mexem na constante associada.)
Alguns textos, como o livro de Sedgewick e Flajolet~\cite[p.~362]{SedgewickFlajolet2013},
chamam a condição acima de ``ordem de heap'';
então, o que eles chamam de ``árvore em ordem de heap``,
nós chamaremos simplesmente de ``heap''.

Ao contrário das árvores binárias,
existem tantas heaps quantas permutações sobre $n$ elementos.

\begin{proposition}
    Existem $n!$ diferentes heaps de $n$ elementos
    cujas prioridades são números do conjunto $\{1, \dots, n\}$.
\end{proposition}

\begin{proof}
    Iremos construir uma bijeção entre os dois conjuntos.
    A figura~\ref{fig:heap-permutation} contém um exemplo deste mapeamento.

    \begin{figure}[h]
        \centering
        \begin{tikzpicture}[binary tree layout, nodes={circle, draw}]
            \node {8}
            child { node {4}
                child { node {2} }
                child { node {3} }
            }
            child { node {7}
                child[missing]
                child { node {6}
                    child { node {1} }
                    child { node {5} }
                }
            };
        \end{tikzpicture}
        \caption{Heap correspondente à permutação $(2, 4, 3, 8, 7, 1, 6, 5)$.}
        \label{fig:heap-permutation}
    \end{figure}

    Dado um heap sobre o conjunto $\{1, \dots, n\}$,
    construa uma permutação
    simplesmente listando as prioridades ao percorrer a árvore em ordem simétrica
    (in-order).
    Dois heaps distintos produzirão permutações diferentes,
    portanto, existem no máximo $n!$ heaps distintas.

    Agora, dada uma permutação $\sigma$,
    escolha $k$ tal que $\sigma(k) = n$; este é o maior valor da permutação.
    Construa um heap colocando o elemento $k$ na raíz,
    e então construa a subárvore esquerda recursivamente com a lista
    $(\sigma(1), \dots, \sigma(k-1))$,
    e a subárvore direita com a lista $(\sigma(k+1), \dots, \sigma(n))$.
    Duas permutações diferentes diferem ou na posição do maior valor,
    e nesse caso as subárvores geradas terão tamanhos diferentes,
    ou em alguma das duas partições,
    e neste caso alguma das subárvores será diferente.
    Portanto, existem ao menos $n!$ heaps distintas.

    Concluímos que existem exatamente $n!$ heaps distintas.
\end{proof}

De posse da definição de heap,
podemos definir treap.

\begin{definition}
    Uma árvore binária cujos nós possuem tanto chaves quanto prioridades associadas
    é uma \emph{treap}
    se for simultaneamente uma árvore de busca de acordo com as chaves
    e um heap de acordo com as prioridades.
\end{definition}

As prioridades serão escolhidas aleatoriamente.

Treaps são, portanto, árvores-heap.
As chaves impõem a ordem dos nós;
como existem $C_n$ diferentes árvores de busca possíveis para a mesma ordem,
as prioridades aleatórias forçam a escolha de uma destas árvores.
Dado um conjunto $X = \{ (k_i, p_i) \mid 1 \leq i \leq n) \}$,
se todas as prioridades forem distintas,
existe uma única treap que contém os elementos deste conjunto.
Para construí-la,
observe que a raíz da árvore necessariamente será o elemento de maior prioridade.
Depois de fixada a raíz,
a exigência do ordenamento determina quais nós ficarão na subárvore esquerda
e quais ficarão na subárvore direita.
Repita esta construção em cada subárvore;
por indução, existe uma única treap em cada caso,
o que garante a existência e unicidade da árvore atual.

Alternativamente,
podemos construir uma árvore binária de busca
inserindo os nós em ordem decrescente de prioridade.
A construção padrão sempre adiciona um novo nó como uma folha,
portanto teremos um heap;
e a ordem é garantida pelo próprio algoritmo.

Este é o ``truque'' das treaps:
ao escolher aleatoriamente uma prioridade para cada elemento,
estamos, efetivamente,
permutando os elementos e então inserindo-os na árvore.
Conforme analisado no teorema~\ref{thm:shuffle-by-sorting},
todas as permutações dos elementos são equiprováveis.
Portanto,
as árvores binárias de busca geradas
seguem a distribuição analisada no teorema~\ref{thm:average-tree-depth};
assim,
sabemos que a profundidade média das treaps será $3 \log_2 n$.

Operações como inserir e remover elementos
podem ser entendidos como apenas alterando o conjunto de elementos $X$;
como a treap é única e as prioridades são aleatórias,
a distribuição das permutações geradas não é alterada,
mantendo a profundidade esperada em $3 \log_2 n$.
(Observe que a inserção de um novo elemento
exige a geração de uma nova prioridade aleatória.)

\subsection{Inserção e remoção}
Esta seção descreve uma implementação de inserção e remoção de valores numa treap.
Os algoritmos estão em pseudocódigo.
Estes algoritmos foram baseados no artigo de Seidel e Aragon~%
\cite[p.~450]{SeidelAragon1996},
que contém uma discussão mais completa sobre aspectos práticos da implementação de treaps.

Ignoraremos tanto a etapa da geração aleatória de prioridades
quanto casos de uso como mapeamentos e filas de prioridade;
portanto,
o nó da treap contém quatro membros:
uma chave, uma prioridade,
e dois ponteiros para as subárvores.

Assumiremos, nesta seção,
que a passagem de parâmetros ocorre por referência.

Em ambos os casos,
trabalharemos recursivamente.
Para a inserção,
após inserir o novo elemento na subárvore esquerda (digamos),
pode ser que a raiz da subárvore esquerda
tenha ficado com prioridade maior do que o nó atual.
Neste caso,
rotacionando o nó atual para a direita promove o outro nó a raiz,
sem alterar a ordem das chaves
(que é preservada pela rotação)
nem das prioridades
(pois apenas um novo nó foi inserido na subárvore esquerda).
A inserção na subárvore direita é simétrica.

\begin{codebox}
    \Procname{$\proc{Treap-Insert}(T, \id{key}, \id{priority})$}
    \li \If $T == \id{null}$
    \li \Then
            $T \gets \proc{New-Node(\id{key}, \id{priority})}$
    \li \ElseIf $\id{key} < \attrib{T}{key}$
    \li \Then
            $\proc{Treap-Insert}(\attrib{T}{lchild}, \id{key}, \id{priority})$
    \li     \If $\attribb{T}{lchild}{priority} > \attrib{T}{priority}$
    \li         \Then $\proc{Rotate-Right}(T)$
            \End
    \li \ElseIf $\attrib{T}{key} < \id{key}$
    \li \Then
            $\proc{Treap-Insert}(\attrib{T}{rchild}, \id{key}, \id{priority})$
    \li     \If $\attribb{T}{rchild}{priority} > \attrib{T}{priority}$
    \li         \Then $\proc{Rotate-Left}(T)$
            \End
    \li \ElseNoIf
    \li     \Comment A chave já aparece na árvore.
        \End
\end{codebox}

Para a remoção,
faremos em duas partes:
primeiro, localizar a subárvore na treap
que possui o elemento a ser removido como raiz.

\begin{codebox}
    \Procname{$\proc{Treap-Remove}(T, \id{key})$}
    \li \If $T == \id{null}$
    \li \Then \Comment Elemento não aparece na árvore.
    \li \ElseIf $\id{key} < \attrib{T}{key}$
    \li \Then $\proc{Treap-Remove}(\attrib{T}{lchild}, \id{key})$
    \li \ElseIf $\attrib{T}{key} < \id{key}$
    \li \Then $\proc{Treap-Remove}(\attrib{T}{rchild}, \id{key})$
    \li \ElseNoIf
    \li     $\proc{Root-Remove}(T)$
        \End
\end{codebox}

E, agora, excluiremos a raiz desta subárvore.
Se algum dos nós filhos da raiz não estiver presente,
basta substituí-lo pelo outro nó (que pode ser nulo);
caso contrário,
rotacionaremos a raiz para baixo
e chamaremos o procedimento recursivamente.
Em algum momento,
o nó a ser removido se tornará folha,
caindo no caso anterior.

\begin{codebox}
    \Procname{$\proc{Root-Remove}(T)$}
    \li \If $\attrib{T}{lchild} == \id{null}$
    \li \Then $T \gets \attrib{T}{rchild}$
    \li \ElseIf $\attrib{T}{rchild} == \id{null}$
    \li \Then $T \gets \attrib{T}{lchild}$
    \li \ElseIf $\attribb{T}{lchild}{priority} < \attribb{T}{rchild}{priority}$
    \li \Then
            $\proc{Rotate-Left}(T)$
    \li     $\proc{Root-Remove}(\attrib{T}{lchild})$
    \li \ElseNoIf
    \li     $\proc{Rotate-Right}(T)$
    \li     $\proc{Root-Remove}(\attrib{T}{rchild})$
        \End
\end{codebox}

\subsection{Teste de performance}
Esta seção relata os resultados obtidos num teste de performance realizados com treaps.
O objetivo era comparar treaps com árvores AVL e árvores RB.

O teste foi escrito em \texttt{C++}.
A implementação de treaps segue a descrita na seção anterior.
Foram usados dois geradores de números aleatórios distintos:
Mersenne Twister e Xorshift.
(O Mersenne Twister é parte da biblioteca padrão do C++.
O Xorshift foi desenvolvido por George Marsaglia \cite{Marsaglia2003}.)
A implementação de árvores AVL segue o que foi visto em aula.
Para árvores RB,
foi usado a classe \texttt{std::set} da biblioteca padrão do \texttt{C++},
que utiliza árvores RB.
Os testes foram executados num processador Intel Core i5-2410M, 2.3 GHz,
sob Ubuntu 15.04 (kernel Linux 3.19).
O programa foi compilado com \texttt{g++ 4.9.2},
com a \texttt{libstdc++} versão \texttt{3.4.19},
usando a flag \texttt{-O3} para otimização.
O código está disponível em \verb"github.com/royertiago/MAC6711-src".

Cada teste foi repetido 12 vezes,
usando sempre a mesma semente para os geradores de números aleatórios.
A média foi calculada ignorando o menor e maior tempo de execução.
Como o objetivo era apenas testar a performance do processo de balanceamento em si,
em todos os casos a chave usada era apenas um \texttt{int}.

Foram feitos três testes: o primeiro levando em conta apenas a inserção,
o segundo levando em conta inserção e busca,
e o terceiro levando em conta as três operações
(inserção, remoção e busca).

\subsubsection{Primeiro teste: Inserção}

Este teste visava principalmente averiguar o tempo de construção da árvore,
para cada tipo de árvore.
Cada estrutura foi testada com $100\,000$ e $1\,000\,000$ valores distintos,
inseridos em ordem aleatória, e em ordem crescente.
Os resultados estão na Tabela~\ref{tab:insert}.
Interessante notar que treaps foram mais rápidas do que as árvores balanceadas
quando os elementos foram inseridos em ordem.

\begin{table}[h]
    \centering
    \begin{tabular}{l c c c c}
        & & & \multicolumn{2}{c}{Treap} \\ \cmidrule{4-5}
            & AVL & RB & MT & Xorshift \\
        \midrule
        $100\,000$ valores aleatórios & 56.1 & 46.3 & 49.9 & 44.2 \\
        $1\,000\,000$ valores aleatórios & 1166.6 & 934.9 & 1306.1 & 1174.8 \\
        $100\,000$ em ordem crescente & 19.3 & 19.8 & 10.6 & 13.7 \\
        $1\,000\,000$ em ordem crescente & 297.5 & 312.1 & 113.4 & 112.7 \\
        \bottomrule
    \end{tabular}
    \caption{Tempo médio (em milissegundos) para a construção da árvore.}
    \label{tab:insert}
\end{table}

\subsubsection{Segundo teste: Inserção e Busca}

Este teste continua o anterior,
estendendo-o com $80\,000$ buscas por elementos que estão na árvore
e $40\,00$ elementos que não estão, para $100\,000$ valores,
e $800\,000$ e $400\,000$, respectivamente, para $1\,000\,000$ valores.
Os resultados estão na Tabela~\ref{tab:insert-remove}.

\begin{table}[h]
    \centering
    \begin{tabular}{l c c c c}
        & & & \multicolumn{2}{c}{Treap} \\ \cmidrule{4-5}
            & AVL & RB & MT & Xorshift \\
        \midrule
        $100\,000$ valores aleatórios & 357.2 & 465.2 & 426.1 & 424.2 \\
        $1\,000\,000$ valores aleatórios & 2108.4 & 1946.9 & 2706.2 & 2686.8 \\
        $100\,000$ em ordem crescente & 331.0 & 501.8 & 442.7 & 513.5 \\
        $1\,000\,000$ em ordem crescente & 1409.1 & 1693.2 & 1270.6 & 1196.0 \\
        \bottomrule
    \end{tabular}
    \caption{Tempo médio (em milissegundos) para a construção e busca.}
    \label{tab:insert-remove}
\end{table}

Aqui, a busca expôs a fraqueza das treaps:
sua altura esperada é aproximadamente $3 \log_2 n$,
muito maior do que o $2\log_2 n$ das árvores RB
e $1.44 \log_2 n$ das árvores AVL.

\subsubsection{Terceiro teste: Inserção, Remoção e Busca}

O teste anterior foi estendido com a remoção de $50\,000$ elementos
para o primeiro caso e $500\,000$ para o segundo.
Foram feitos dois cenários distintos.
No primeiro, todos os valores são inseridos de uma vez só,
depois todos os valores que precisam ser removidos o são,
e só então são efetuadas as buscas.
No segundo cenário, após a inserção de metade dos valores,
as demais operações se deram em sequência aleatória.
Os resultados estão na Tabela~\ref{tab:mixed}.

\begin{table}[h]
    \centering
    \begin{tabular}{l c c c c}
        & & & \multicolumn{2}{c}{Treap} \\ \cmidrule{4-5}
            & AVL & RB & MT & Xorshift \\
        \midrule
        $270\,000$ operações, em sequência & 112.4 & 110.6 & 105.3 & 91.9 \\
        $2\,700\,000$ operações, em sequência & 2586.8 & 2246.1 & 3260.0 & 2754.9 \\
        $270\,000$ operações, aleatórias & 71.3 & 69.1 & 70.4 & 70.3 \\
        $2\,700\,000$ operações, aleatórias & 2897.0 & 2420.0 & 2774.9 & 2648.3 \\
        \bottomrule
    \end{tabular}
    \caption{Tempo médio (em milissegundos) para a construção e busca.}
    \label{tab:mixed}
\end{table}

Com $1\,000\,000$ elementos,
treaps foram mais lentas para a inserção em sequência.
Mas, para operações aleatórias,
treaps foram mais rápidas do que as árvores AVL.
Isso não foi suficiente para superar a implementação de árvores RB da \texttt{libstdc++}.
Estas observações são contrárias às encontradas por Heger \cite[p.65]{Heger2004};
possivelmente,
uma implementação mais cuidadosa de treaps
possa superar o desempenho de árvores RB.


\bibliographystyle{plain}
\bibliography{bib/bibliography}

\end{document}
