\documentclass{article}
\usepackage[utf8]{inputenc}
\usepackage[T1]{fontenc}

\usepackage[brazil]{babel}

\usepackage{amsmath}
\usepackage{amssymb}
\usepackage{amsthm}

\newtheorem{theorem}{Teorema}[section]
\newtheorem{lemma}[theorem]{Lema}
\newtheorem{corollary}[theorem]{Corolário}
\newtheorem{proposition}[theorem]{Proposição}

\theoremstyle{definition}
\newtheorem{definition}[theorem]{Definição}

\theoremstyle{remark}
\newtheorem*{notation}{Notação}

\usepackage{tikz}
\usetikzlibrary{graphs}
\usetikzlibrary{graphdrawing}
\usegdlibrary{trees}

\begin{document}

\title{Árvores de Busca Binária}
\author{Tiago Royer}
\date{28 de abril de 2016}
\maketitle

\begin{abstract}
    Este artigo discute algumas propriedades de árvores de busca binária
    (como, por exemplo, altura média de uma árvore aleatória).
\end{abstract}

\section{Introdução e notação usada}

\begin{definition}
    Uma \emph{árvore binária}
    é ou um único nó externo,
    ou um nó interno,
    que é uma estrutura que contém duas outras árvores
    (a subárvore esquerda e a subárvore direita)
    \cite[p.~257]{SedgewickFlajolet2013}.
\end{definition}

Árvores binárias são o mais simples tipo de árvore
estudados pela Ciência da Computação.
Nas aplicações,
tipicamente ``penduramos'' alguma informação nos nós internos da árvore
e usamos a estrutura da árvore para impor uma relação entre essas informações.
Os nós externos costumam não carregar informação:
eles são usados apenas para simplificar a definição de ``árvore''
--- por exemplo,
em vez de dizer que certo nodo ``não possui uma subárvore esquerda'',
nós dizemos que a subárvore esquerda é um nó externo.
Podemos, por exemplo,
implementar o nó externo como \texttt{null}.
Neste artigo,
desprezaremos a existência desses nós externos
e nos referiremos aos nós internos simplesmente por ``nó'' (ou ``nodo'').
Assim, por exemplo,
uma árvore que só contém um único nó externo
será considerada ``vazia'', ou ``inexistente'' se for subárvore.
Também os excluiremos das contagens;
portanto, consideraremos que
uma árvore que é um nó interno cujas duas subárvores são nós externos
tem apenas um nó.

\begin{definition}
    Uma \emph{árvore binária de busca}
    é uma árvore binária em que
    todos os nós estão associados a alguma chave,
    de forma que a chave de um nó
    é maior que todos os nós de sua subárvore esquerda
    e menor que todos os nós de sua subárvore direita
    \cite[p.~282]{SedgewickFlajolet2013}.
\end{definition}

Existe um procedimento padrão para gerar uma árvore binária de busca
a partir de uma lista de números.
A ideia é inserir os números na ordem em que aparecem,
sempre procurando o ``lugar certo'' na árvore para a inserção.
A figura~\ref{fig:bst-construction} ilustra o processo para a lista $(4, 5, 2, 6, 1, 3)$.

\begin{figure}[h]
    \centering
    \begin{tikzpicture}[binary tree layout, nodes={circle, draw}]
        \node {4};
        \node {4}
        child[missing]
        child { node {5} };
        \node {4}
        child { node {2} }
        child { node {5} };
        \node {4}
        child { node {2} }
        child { node {5}
            child[missing]
            child { node {6} }
        };
        \node {4}
        child { node {2}
            child { node {1} }
        }
        child { node {5}
            child[missing]
            child { node {6} }
        };
        \node {4}
        child { node {2}
            child { node {1} }
            child { node {3} }
        }
        child { node {5}
            child[missing]
            child { node {6} }
        };
    \end{tikzpicture}
    \caption{
        Construção de uma árvore binária de busca a partir da lista $(4, 5, 2, 6, 1, 3)$.
    }
    \label{fig:bst-construction}
\end{figure}

Pela definição,
uma árvore binária de busca não contém chaves repetidas.
Além disso,
para os propósitos deste artigo,
o valor real das chaves não importa
--- o que é relevante é a \emph{ordem} entre eles.
Portanto,
podemos assumir que,
numa árvore com $n$ nós,
seus elementos são os números $1, \dots, n$.
Assim,
a lista que der origem a uma árvore binária de busca
será uma permutação.

Neste artigo,
nos concentraremos nas árvores binárias de busca
e seu contraste estatístico com as árvores binárias uniformes.

\begin{notation}
    Como frequentemente interpretaremos a permutação como uma lista,
    identificaremos as permutações pela lista de seus elementos;
    por exemplo, se $\sigma = (1, 3, 2)$,
    então $\sigma$ é a permutação que satisfaz
    $\sigma(1) = 1, \sigma(2) = 3, \sigma(3) = 2$.

    O conjunto de todas as permutações de $n$ elementos será denotado por $S_n$.
    A permutação vazia, sobre $0$ elementos, é válida;
    assim, $|S_0| = 1$.
    O conjunto de todas as permutações, $\bigcup_{i \in \mathbb N} S_i$,
    será denotado por $\mathcal S$.
\end{notation}

\section{Propriedades Estatísticas de Árvores Binárias Aleatórias}

\subsection{Altura média de uma árvore de busca binária}

(A demonstração construída nesta seção é adaptada do livro de Cormen et al.%
~\cite[p.~300]{CormenLeisersonRivestStein2009}.)

Na construção de uma árvore binária de busca
a partir de uma permutação de $(1, \dots, n)$,
após adicionamos o primeiro número como a raíz,
a adição dos próximos elementos colocará
todos os números menores que a raíz na subárvore à esquerda
e todos os números maiores que a raíz na subárvore à direita.
Na análise, portanto,
precisaremos particionar uma permutação nestes dois subconjuntos.
Assim,
antes de começarmos a conta,
criaremos uma notação para esta operação
e provaremos dois lemas que serão usados para facilitar sua manipulação.

Por exemplo, se a permutação $\sigma$ corresponde a lista $(4, 5, 3, 6, 1, 2)$,
ao construírmos uma árvore a partir de $\sigma$,
$\sigma(1) = 4$ será a raíz,
a subárvore à esquerda será a árvore construída a partir da lista $(3, 1, 2)$,
e a subárvore à direita, da lista $(5, 6)$.
Como o valor dos nós da árvore não é relevante (apenas seu valor relativo é),
podemos supor que a subárvore à direita foi gerada pela lista $(1, 2)$,
que é a mesma lista,
mas ``normalizada'' para o intervalo $[1, 2]$.
Chamaremos a lista $(3, 1, 2)$,
a parte que vai para a esquerda, de $\sigma_<$
(é a parte de $\sigma$ que é menor que $\sigma(1)$, a raíz),
e a lista $(1, 2)$,
a que foi gerada a partir da lista $(5, 6)$, de $\sigma_>$.

A definição a seguir formaliza estes nomes.

\begin{definition}
    Seja $\sigma \in S_n$ uma permutação.
    Defina $\sigma_<$ e $\sigma_>$
    como sendo a sequência dos elementos menores e maiores que $\sigma(1)$
    (o primeiro elemento)
    respectivamente,
    normalizados para os intervalo $\{1, \sigma(1) - 1\}$ e $\{1, n - \sigma(1)\}$,
    respectivamente.
    Isto é,
    se $\{a_i\}_{i = 1}^{\sigma(1)-1}$ é a sequência crescente
    dos índices satisfazendo $\sigma(a_i) < \sigma(1)$,
    então $\sigma_< \in S_{\sigma(1) - 1}$
    é a permutação que satisfaz $\sigma_<(i) = \sigma(a_i)$ para todo $i$;
    e se $\{b_i\}_{i = 1}^{n - \sigma(1)}$ é a sequência crescente
    dos índices satisfazendo $\sigma(b_i) > \sigma(1)$,
    então $\sigma_> \in S_{n - \sigma(1)}$
    é a permutação que satisfaz $\sigma_>(i) = \sigma(b_i) - \sigma(1)$ para todo $i$.
\end{definition}

A normalização que $\sigma_>$ sofre garante que
tanto $\sigma_<$ quanto $\sigma_>$ sejam permutações.
Como usaremos ambas,
é importante sabermos como converter somatórios sobre um tipo no outro.

\begin{lemma}
    Chame de $\mathcal S$ o conjunto de todas as permutações.
    Se $f: \mathcal S \to \mathbb N$ é uma função qualquer,
    então
    \begin{equation}
        \sum_{\sigma \in S_n} f(\sigma_<) = \sum_{\sigma \in S_n} f(\sigma_>).
        \label{eq:lower-permutation-to-upper}
    \end{equation}
\end{lemma}

\begin{proof}
    Se $\sigma \in S_n$,
    defina $\overline \sigma \in S_n$
    (o ``conjugado'' de $\sigma$)
    por $\overline \sigma(i) = n + 1 - \sigma(i)$.
    Esta operação é bijetora
    e inverte a relação de ordem entre os elementos de $\sigma$.
    Dessa forma, os elementos maiores que $\sigma(1)$, em $\sigma$,
    se tornarão menores que $\overline \sigma(1)$, em $\overline \sigma$.
    Portanto,
    os elementos de $(\overline \sigma)_<$ são,
    exatamente,
    os conjugados dos elementos de $\sigma_>$;
    portanto, temos
    \begin{equation}
        (\overline \sigma)_< = \overline{(\sigma_>)}.
        \label{eq:conjugate-permutation}
    \end{equation}
    Portanto,
    \begin{align*}
        \sum_{\sigma \in S_n} f(\sigma_<) &= \sum_{\sigma \in S_n} f((\overline \sigma)_<)
        &\text{pois $\sigma \mapsto \overline \sigma$ é uma bijeção} \\
        &= \sum_{\sigma \in S_n} f(\overline{(\sigma_>)})
        &\text{pela equação~\ref{eq:conjugate-permutation}} \\
        &= \sum_{\sigma \in S_n} f(\sigma_>)
        &\text{pois $\sigma \mapsto \overline \sigma$ é uma bijeção.}
    \end{align*}
\end{proof}


\bibliographystyle{plain}
\bibliography{bib/bibliography}

\end{document}
