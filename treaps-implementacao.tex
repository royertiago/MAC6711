\subsection{Inserção e remoção}
Esta seção descreve uma implementação de inserção e remoção de valores numa treap.
Os algoritmos estão em pseudocódigo.
Estes algoritmos foram baseados no artigo de Seidel e Aragon~%
\cite[p.~450]{SeidelAragon1996},
que contém uma discussão mais completa sobre aspectos práticos da implementação de treaps.

Ignoraremos tanto a etapa da geração aleatória de prioridades
quanto casos de uso como mapeamentos e filas de prioridade;
portanto,
o nó da treap contém quatro membros:
uma chave, uma prioridade,
e dois ponteiros para as subárvores.

Assumiremos, nesta seção,
que a passagem de parâmetros ocorre por referência.

Em ambos os casos,
trabalharemos recursivamente.
Para a inserção,
após inserir o novo elemento na subárvore esquerda (digamos),
pode ser que a raiz da subárvore esquerda
tenha ficado com prioridade maior do que o nó atual.
Neste caso,
rotacionando o nó atual para a direita promove o outro nó a raiz,
sem alterar a ordem das chaves
(que é preservada pela rotação)
nem das prioridades
(pois apenas um novo nó foi inserido na subárvore esquerda).
A inserção na subárvore direita é simétrica.

\begin{codebox}
    \Procname{$\proc{Treap-Insert}(T, \id{key}, \id{priority})$}
    \li \If $T == \id{null}$
    \li \Then
            $T \gets \proc{New-Node(\id{key}, \id{priority})}$
    \li \ElseIf $\id{key} < \attrib{T}{key}$
    \li \Then
            $\proc{Treap-Insert}(\attrib{T}{lchild}, \id{key}, \id{priority})$
    \li     \If $\attribb{T}{lchild}{priority} > \attrib{T}{priority}$
    \li         \Then $\proc{Rotate-Right}(T)$
            \End
    \li \ElseIf $\attrib{T}{key} < \id{key}$
    \li \Then
            $\proc{Treap-Insert}(\attrib{T}{rchild}, \id{key}, \id{priority})$
    \li     \If $\attribb{T}{rchild}{priority} > \attrib{T}{priority}$
    \li         \Then $\proc{Rotate-Left}(T)$
            \End
    \li \ElseNoIf
    \li     \Comment A chave já aparece na árvore.
        \End
\end{codebox}

Para a remoção,
faremos em duas partes:
primeiro, localizar a subárvore na treap
que possui o elemento a ser removido como raiz.

\begin{codebox}
    \Procname{$\proc{Treap-Remove}(T, \id{key})$}
    \li \If $T == \id{null}$
    \li \Then \Comment Elemento não aparece na árvore.
    \li \ElseIf $\id{key} < \attrib{T}{key}$
    \li \Then $\proc{Treap-Remove}(\attrib{T}{lchild}, \id{key})$
    \li \ElseIf $\attrib{T}{key} < \id{key}$
    \li \Then $\proc{Treap-Remove}(\attrib{T}{rchild}, \id{key})$
    \li \ElseNoIf
    \li     $\proc{Root-Remove}(T)$
        \End
\end{codebox}

E, agora, excluiremos a raiz desta subárvore.
Se algum dos nós filhos da raiz não estiver presente,
basta substituí-lo pelo outro nó (que pode ser nulo);
caso contrário,
rotacionaremos a raiz para baixo
e chamaremos o procedimento recursivamente.
Em algum momento,
o nó a ser removido se tornará folha,
caindo no caso anterior.

\begin{codebox}
    \Procname{$\proc{Root-Remove}(T)$}
    \li \If $\attrib{T}{lchild} == \id{null}$
    \li \Then $T \gets \attrib{T}{rchild}$
    \li \ElseIf $\attrib{T}{rchild} == \id{null}$
    \li \Then $T \gets \attrib{T}{lchild}$
    \li \ElseIf $\attribb{T}{lchild}{priority} < \attribb{T}{rchild}{priority}$
    \li \Then
            $\proc{Rotate-Left}(T)$
    \li     $\proc{Root-Remove}(\attrib{T}{lchild})$
    \li \ElseNoIf
    \li     $\proc{Rotate-Right}(T)$
    \li     $\proc{Root-Remove}(\attrib{T}{rchild})$
        \End
\end{codebox}
