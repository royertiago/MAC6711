\section{Introdução e notação usada}

\begin{definition}
    Uma \emph{árvore binária}
    é ou um único nó externo,
    ou um nó interno,
    que é uma estrutura que contém duas outras árvores
    (a subárvore esquerda e a subárvore direita)
    \cite[p.~257]{SedgewickFlajolet2013}.
\end{definition}

Árvores binárias são o mais simples tipo de árvore
estudados pela Ciência da Computação.
Nas aplicações,
tipicamente ``penduramos'' alguma informação nos nós internos da árvore
e usamos a estrutura da árvore para impor uma relação entre essas informações.
Os nós externos costumam não carregar informação:
eles são usados apenas para simplificar a definição de ``árvore''
--- por exemplo,
em vez de dizer que certo nodo ``não possui uma subárvore esquerda'',
nós dizemos que a subárvore esquerda é um nó externo.
Podemos, por exemplo,
implementar o nó externo como \texttt{null}.
Neste artigo,
desprezaremos a existência desses nós externos
e nos referiremos aos nós internos simplesmente por ``nó'' (ou ``nodo'').
Assim, por exemplo,
uma árvore que só contém um único nó externo
será considerada ``vazia'', ou ``inexistente'' se for subárvore.
Também excluiremos-nos das contagens;
portanto, consideraremos que
uma árvore que é um nó interdo cujas duas subárvores são nós externos
tem apenas um nó.
