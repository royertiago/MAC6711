\section{Introdução e notação usada}

\begin{definition}
    Uma \emph{árvore binária}
    é ou um único nó externo,
    ou um nó interno,
    que é uma estrutura que contém duas outras árvores
    (a subárvore esquerda e a subárvore direita)
    \cite[p.~257]{SedgewickFlajolet2013}.
\end{definition}

Árvores binárias são o mais simples tipo de árvore
estudados pela Ciência da Computação.
Nas aplicações,
tipicamente ``penduramos'' alguma informação nos nós internos da árvore
e usamos a estrutura da árvore para impor uma relação entre essas informações.
Os nós externos costumam não carregar informação:
eles são usados apenas para simplificar a definição de ``árvore''
--- por exemplo,
em vez de dizer que certo nodo ``não possui uma subárvore esquerda'',
nós dizemos que a subárvore esquerda é um nó externo.
Podemos, por exemplo,
implementar o nó externo como \texttt{null}.
Neste artigo,
desprezaremos a existência desses nós externos
e nos referiremos aos nós internos simplesmente por ``nó'' (ou ``nodo'').
Assim, por exemplo,
uma árvore que só contém um único nó externo
será considerada ``vazia'', ou ``inexistente'' se for subárvore.
Também excluiremos-nos das contagens;
portanto, consideraremos que
uma árvore que é um nó interdo cujas duas subárvores são nós externos
tem apenas um nó.

\begin{definition}
    Uma \emph{árvore binária de busca}
    é uma árvore binária em que
    todos os nós estão associados a alguma chave,
    de forma que a chave de um nó
    é maior que todos os nós de sua subárvore esquerda
    e menor que todos os nós de sua subárvore direita
    \cite[p.~282]{SedgewickFlajolet2013}.
\end{definition}

Existe um procedimento padrão para gerar uma árvore binária de busca
a partir de uma lista de números.
A ideia é inserir os números na ordem em que aparecem,
sempre procurando o ``lugar certo'' na árvore para a inserção.
A figura~\ref{fig:bst-construction} ilustra o processo para a lista $(4, 5, 2, 6, 1, 3)$.

\begin{figure}[h]
    \centering
    \begin{tikzpicture}[binary tree layout, nodes={circle, draw}]
        \node {4};
        \node {4}
        child[missing]
        child { node {5} };
        \node {4}
        child { node {2} }
        child { node {5} };
        \node {4}
        child { node {2} }
        child { node {5}
            child[missing]
            child { node {6} }
        };
        \node {4}
        child { node {2}
            child { node {1} }
        }
        child { node {5}
            child[missing]
            child { node {6} }
        };
        \node {4}
        child { node {2}
            child { node {1} }
            child { node {3} }
        }
        child { node {5}
            child[missing]
            child { node {6} }
        };
    \end{tikzpicture}
    \caption{
        Construção de uma árvore binária de busca a partir da lista $(4, 5, 2, 6, 1, 3)$.
    }
    \label{fig:bst-construction}
\end{figure}

Pela definição,
uma árvore binária de busca não contém chaves repetidas.
Além disso,
para os propósitos deste artigo,
o valor real das chaves não importa
--- o que é relevante é a \emph{ordem} entre eles.
Portanto,
podemos assumir que,
numa árvore com $n$ nós,
seus elementos são os números $1, \dots, n$.
Assim,
a lista que der origem a uma árvore binária de busca
será uma permutação.

Neste artigo,
nos concentraremos nas árvores binárias de busca
e seu contraste estatístico com as árvores binárias uniformes.
